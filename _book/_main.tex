% Options for packages loaded elsewhere
\PassOptionsToPackage{unicode}{hyperref}
\PassOptionsToPackage{hyphens}{url}
%
\documentclass[
  oneside]{book}
\usepackage{amsmath,amssymb}
\usepackage{lmodern}
\usepackage{iftex}
\ifPDFTeX
  \usepackage[T1]{fontenc}
  \usepackage[utf8]{inputenc}
  \usepackage{textcomp} % provide euro and other symbols
\else % if luatex or xetex
  \usepackage{unicode-math}
  \defaultfontfeatures{Scale=MatchLowercase}
  \defaultfontfeatures[\rmfamily]{Ligatures=TeX,Scale=1}
\fi
% Use upquote if available, for straight quotes in verbatim environments
\IfFileExists{upquote.sty}{\usepackage{upquote}}{}
\IfFileExists{microtype.sty}{% use microtype if available
  \usepackage[]{microtype}
  \UseMicrotypeSet[protrusion]{basicmath} % disable protrusion for tt fonts
}{}
\makeatletter
\@ifundefined{KOMAClassName}{% if non-KOMA class
  \IfFileExists{parskip.sty}{%
    \usepackage{parskip}
  }{% else
    \setlength{\parindent}{0pt}
    \setlength{\parskip}{6pt plus 2pt minus 1pt}}
}{% if KOMA class
  \KOMAoptions{parskip=half}}
\makeatother
\usepackage{xcolor}
\usepackage[margin=2.5cm]{geometry}
\usepackage{longtable,booktabs,array}
\usepackage{calc} % for calculating minipage widths
% Correct order of tables after \paragraph or \subparagraph
\usepackage{etoolbox}
\makeatletter
\patchcmd\longtable{\par}{\if@noskipsec\mbox{}\fi\par}{}{}
\makeatother
% Allow footnotes in longtable head/foot
\IfFileExists{footnotehyper.sty}{\usepackage{footnotehyper}}{\usepackage{footnote}}
\makesavenoteenv{longtable}
\usepackage{graphicx}
\makeatletter
\def\maxwidth{\ifdim\Gin@nat@width>\linewidth\linewidth\else\Gin@nat@width\fi}
\def\maxheight{\ifdim\Gin@nat@height>\textheight\textheight\else\Gin@nat@height\fi}
\makeatother
% Scale images if necessary, so that they will not overflow the page
% margins by default, and it is still possible to overwrite the defaults
% using explicit options in \includegraphics[width, height, ...]{}
\setkeys{Gin}{width=\maxwidth,height=\maxheight,keepaspectratio}
% Set default figure placement to htbp
\makeatletter
\def\fps@figure{htbp}
\makeatother
\usepackage[normalem]{ulem}
\setlength{\emergencystretch}{3em} % prevent overfull lines
\providecommand{\tightlist}{%
  \setlength{\itemsep}{0pt}\setlength{\parskip}{0pt}}
\setcounter{secnumdepth}{5}
\usepackage{booktabs}
\usepackage{fontspec}
\usepackage{multirow}
\usepackage{multicol}
\usepackage{colortbl}
\usepackage{hhline}
\usepackage{longtable}
\usepackage{array}
\usepackage{hyperref}
\ifLuaTeX
  \usepackage{selnolig}  % disable illegal ligatures
\fi
\usepackage[]{natbib}
\bibliographystyle{plainnat}
\IfFileExists{bookmark.sty}{\usepackage{bookmark}}{\usepackage{hyperref}}
\IfFileExists{xurl.sty}{\usepackage{xurl}}{} % add URL line breaks if available
\urlstyle{same} % disable monospaced font for URLs
\hypersetup{
  pdftitle={Draft Panduan Survei Keanekaragaman Hayati},
  pdfauthor={Biodive FFI`s IP},
  hidelinks,
  pdfcreator={LaTeX via pandoc}}

\title{Draft Panduan Survei Keanekaragaman Hayati}
\author{Biodive FFI`s IP}
\date{2022-09-29}

\begin{document}
\maketitle

{
\setcounter{tocdepth}{4}
\tableofcontents
}
\hypertarget{prakata}{%
\chapter*{Prakata}\label{prakata}}
\addcontentsline{toc}{chapter}{Prakata}

\textbf{Apa dan untuk siapa panduan ini?}

Panduan ini ditujukan bagi siapa saja yang berminat untuk melakukan survei keanekararagaman hayati (kehati) terutama bagi praktisi di ruang lingkup Fauna \& Flora International -- Indonesia Programme (FFI`s IP). Panduan ini disusun supaya pemantauan kehati dapat dilaksanakan dengan standar minimum yang sama, akurat dan dapat digunakan untuk pengambilan keputusan secara ilmiah.

Panduan ini dibuat sebagai ringkasan secara umum untuk melakukan pemantauan pada 4 taksa sebagai berikut; Avifauna, Herpetofauna, Mamalia serta Vegetasi yang berada dalam bioma terestrial. Panduan ini memiliki beberapa asumsi yang harus dipenuhi serta berbagai keterbatasan, disesuaikan dengan target dan luaran dari survei itu sendiri.

Kami menyadari bahwasannya metode pemantauan kehati selalu berkembang sehingga timbal balik dari pembaca diharapkan dapat terus menyempurnakan kebutuhan yang sesuai bagi para praktisi konservasi yang menggunakan panduan ini.

\hypertarget{pendahuluan}{%
\chapter*{Pendahuluan}\label{pendahuluan}}
\addcontentsline{toc}{chapter}{Pendahuluan}

Survei keanekaragaman hayati, bertujuan untuk mendapatkan informasi mengenai keberadaan satwa liar pada ruang dan waktu tertentu. Secara umum, survei kehati memiliki dua luaran yaitu;

\begin{enumerate}
\def\labelenumi{(\arabic{enumi})}
\item
  Inventarisasi, yang bertujuan untuk mendapatkan informasi mengenai fauna dan flora pada suatu area. Luarannya digunakan untuk membangun data dasar (baseline data). Biasanya, dalam kegiatan ini kita hanya membutuhkan konfirmasi apakah suatu spesies berhasil teridentifikasi di suatu area atau tidak.
\item
  Monitoring, yang dilakukan lebih dari satu kali dalam tahun atau musim yang berbeda untuk mendeteksi adanya perubahan (atau tidak ada perubahan) dalam suatu komunitas biologi. Monitoring juga digunakan untuk dapat melihat efek dari suatu kegiatan (contoh; {[}i{]} perambahan kawasan terhadap komunitas burung liar, {[}ii{]} pencemaran sungai terhadap mortalitas amfibi). Dalam monitoring, biasanya dibutuhkan penilaian kuantitatif yang kuat, daripada sekedar konfirmasi keberadaan spesies.
\end{enumerate}

Dalam kondisi aktual, sangat sulit untuk bisa mendapatkan nilai atau jumlah dari spesies yang seluruhnya menghuni suatu kawasan, terlebih di hutan tropis yang lebat dengan tingkat visibiltas yang rendah. Oleh karena itu, kita harus memiliki rancangan survei yang tepat, supaya mendapatkan sampel yang representatif dari kawasan tersebut \protect\hyperlink{rancangan-survei}{(Bab Rancangan Survei)}. Sindrom data ``\emph{sampah masuk, sampah keluar}'' juga berlaku untuk survei kehati. Jika kualitas data yang dikumpulkan lemah dan rancangan surveinya kurang menggambarkan areal terkait, maka akan sulit untuk menganalisa dan menginterpretasi data. Prosedur pengambilan sampel harus mengikuti protokol rancangan survei serta protokol pengambilan data dilapangan yang kuat, untuk memastikan pengumpulan data yang konsisten dengan kualitas semaksimal mungkin \protect\hyperlink{protokol-survei}{(Bab Prokotol Survei)}.

Menganalisa data merupakan bagian mendasar dari survei pada saat mempersiapkan rancangan survei, konsultasi dengan seorang ahli statistik satwa liar yang berpengalaman akan sangat bermanfaat untuk mempersiapkan analisa yang tepat \protect\hyperlink{analisa-data}{(Bab Analisa Data)}. Selain itu, luaran survei seringkali menjadi laporan kepada donor atau publikasi ilmiah sehingga data-data yang sudah didapatkan menjadi sangat berharga dan diperlukan untuk pengambilan keputusan yang tepat, serta memastikan data yang sudah didapat masih relevan untuk digunakan hingga bertahun-tahun kedepan. Oleh karena itu pengarsipan data juga bagian yang sangat penting, dan akan dibahas pada bagian pengelolaan data kehati \protect\hyperlink{pengelolaan-data}{(Bab Pengelolaan Data)}.

\hypertarget{pra-survei}{%
\chapter*{Pra-Survei}\label{pra-survei}}
\addcontentsline{toc}{chapter}{Pra-Survei}

\hypertarget{keahlian-dasar}{%
\section*{Keahlian Dasar}\label{keahlian-dasar}}
\addcontentsline{toc}{section}{Keahlian Dasar}

\hypertarget{keselamatan-kerja}{%
\section*{Keselamatan Kerja}\label{keselamatan-kerja}}
\addcontentsline{toc}{section}{Keselamatan Kerja}

Survei kehati seringkali dilakukan di daerah terisolir, jauh dari sarana umum dan kebutuhan akan bantuan medis profesional sulit dijangkau. Oleh karena itu penting untuk selalu sadar mengenai bahaya yang mengintai setiap saat, sehingga kita harus selalu waspada selama berkegiatan. Dibawah ini merupakan beberapa tips untuk dapat diikuti.

\textbf{Selalu bekerja bersama tim.} Jangan pernah melakukan pengamatan sendirian, pastikan minimal ada satu anggota lain yang ikut. Jika terjadi kecelakaan kerja, atau kondisi darurat, akan ada rekan kerja yang dapat memberikan pertolongan. Keberadaan rekan kerja juga mengurangi resiko tersesat saat pengamatan.

\textbf{Memberikan rencana perjalanan di luar tim.} Pastikan rekan kerja selain orang di luar tim survei, tahu rencana perjalanan dan kapan kalian akan kembali. Mereka dapat memberikan pertimbangan untuk melakukan evakuasi, jika kalian belum kembali dari waktu yang sudah direncakan. Rencana ini lebih baik jika ditulis detil hari per hari, sehingga mereka tahu perkiraan anda berada dimana pada suatu tanggal spesifik.

\textbf{Persiapkan peralatan keselamatan dengan seksama.} Pastikan membawa peta, kompas dan GPS jika ingin melakukan pengamatan di luar jalur. Membawa senter dan alat penerang jika estimasi perjalanan kalian hingga malam. Membawa peralatan pertolongan pertama jika berjalan jauh dari kamp utama. Membawa suplai makanan ekstra untuk melalui medan yang belum dikenal jika terdapat kelebihan hari.

\textbf{Persiapan untuk dapat memberikan pertolongan pertama.} Ada baiknya seluruh anggota tim, dilatih untuk dapat memberikan pertolongan pertama dengan benar dari pelaku medis profesional setempat (Dokter, petugas puskesmas, KSR PMI dll). Sehingga mereka sudah siap memberikan pertolongan kepada siapapun yang membutuhkan. Peralatan pertolongan pertama yang akan dibawa setidaknya mencakup; seperangkat peralatan penutup luka, tablet antibiotik, tablet malaria, bubuk rehidrasi, salep atau bedak anti jamur dan anti gatal, salep luka bakar dan \emph{snake bite-kit}.

\textbf{Persiapkan jalur evakuasi.} Persiapkan rute evakuasi, seperti titik evakuasi terdekat, kendaraan yang sudah siap sedia untuk menjemput tim yang perlu dievakuasi, sarana medis yang akan digunakan, protokol komunikasi dll.

\textbf{Menggunakan tenaga lokal.} Masyarakat lokal yang memiliki rutinitas berkegiatan di hutan dapat memberikan masukan mengenai jalur yang akan digunakan dan akan lebih waspada terhadap kondisi di kawasan tersebut (Potensi pohon tumbang, area rawan longsor dll).

\textbf{Hindari organisme dan area berbahaya.} Meskipun tujuan survei adalah untuk kajian jenis-jenis ular sekalipun, hindari menangkap ular berbisa. Jika kalian ragu apakah hewan tersebut berbahaya atau tidak sebaikmya tetap dihindari. Beberapa tumbuhan juga dapat menyebabkan gatal-gatal seperti jenis-jenis jelatang (Dendrocnide sp), beberapa lebah ada yang membuat sarang disemak-semak yang apabila tersentuh akan menyerang. Semakin sering ke lapangan anda akan dapat menghindari kejadian-kejadian tersebut. Hindari mandi di area sungai yang berpotensi dihuni oleh buaya. Jangan membuat kamp diarea yang terdapat pohon mati atau berpotensi rubuh.
Perhatikan penggunaan pisau tebas. Penggunaan pisau tebas saat membuka jalur seringkali melukai diri sendiri dan orang didekatnya. Selalu waspada dalam penggunaan pisau tebas, terlebih jika area yang dibuka merupakan semak-semak yang memiliki tingkat kekerasan dan kerapatan yang variatif.

\textbf{Kebersihan adalah prioritas.} Luka-luka kecil akibat duri atau gesekan kayu bisa memberikan infeksi jika tidak rutin dibersihkan dengan air dan sabun antiseptik. Kondisi dapur di area kamp juga harus diperhatikan, terkadang anggota tim membuang sampah organik sembarangan, hal ini dapat menyebabkan masalah pencernaan.

\textbf{TERPENTING. \emph{Use common sense!}.} Seringkali kecelekaan terjadi karena hal yang dari awal dapat dihindari seperti menyebrang sungai deras tanpa pengaman, memanjat pohon untuk mencari sinyal, tersesat karena panik, memanjat tebing terjal tanpa bantuan tali dan lain-lain. Selalu ketahui batas diri masing-masing, anggota tim yang lain mungkin dapat melompat diantara celah tebing dengan mudah, atau melewati tebing hanya dengan menyebrangi sebatang kayu yang dijadikan jembatan, namun anda belum tentu dapat melewatinya. Merupakan pilihan yang bijak untuk berpikir mengenai keselamatan diri dan membuat alternatif pilihan terhadap hambatan yang ditemui.

\hypertarget{pembuatan-transek}{%
\section*{Pembuatan Transek}\label{pembuatan-transek}}
\addcontentsline{toc}{section}{Pembuatan Transek}

\hypertarget{rancangan-survei}{%
\chapter*{Rancangan Survei}\label{rancangan-survei}}
\addcontentsline{toc}{chapter}{Rancangan Survei}

Beragam teknik dalam mencuplik satwa liar sudah banyak dibahas dalam berbagai panduan (Bookhout, 1994; Elzinga et al., 2009). Dalam panduan ini kami hanya menyimpulkan sebagian yang sering dipergunakan dalam ruang lingkup kerja FFI`s IP. Secara umum teknik ini dibagi dalam dua hal dalam koleksi datanya, yaitu secara observatif atau perjumpaan langsung dan penangkapan. Konteks penangkapan dalam hal ini tidak hanya terbatas menangkap satwanya tapi juga dalam media gambar dan suara (tabel \ref{tab:tab1}).

\providecommand{\docline}[3]{\noalign{\global\setlength{\arrayrulewidth}{#1}}\arrayrulecolor[HTML]{#2}\cline{#3}}

\setlength{\tabcolsep}{2pt}

\renewcommand*{\arraystretch}{1.5}

\begin{longtable}[c]{|p{1.23in}|p{2.18in}|p{3.86in}|p{2.39in}}

\caption{Ragam metode survei dan target organisme
}\label{tab:tab1}\\

\hhline{>{\arrayrulecolor[HTML]{666666}\global\arrayrulewidth=2pt}->{\arrayrulecolor[HTML]{666666}\global\arrayrulewidth=2pt}->{\arrayrulecolor[HTML]{666666}\global\arrayrulewidth=2pt}->{\arrayrulecolor[HTML]{666666}\global\arrayrulewidth=2pt}-}

\multicolumn{1}{!{\color[HTML]{000000}\vrule width 0pt}>{\raggedright}p{\dimexpr 1.23in+0\tabcolsep+0\arrayrulewidth}}{\fontsize{11}{11}\selectfont{\textcolor[HTML]{000000}{\global\setmainfont{Arial}{\textbf{Teknik}}}}} & \multicolumn{1}{!{\color[HTML]{000000}\vrule width 0pt}>{\raggedright}p{\dimexpr 2.18in+0\tabcolsep+0\arrayrulewidth}}{\fontsize{11}{11}\selectfont{\textcolor[HTML]{000000}{\global\setmainfont{Arial}{\textbf{Metode}}}}} & \multicolumn{1}{!{\color[HTML]{000000}\vrule width 0pt}>{\raggedright}p{\dimexpr 3.86in+0\tabcolsep+0\arrayrulewidth}}{\fontsize{11}{11}\selectfont{\textcolor[HTML]{000000}{\global\setmainfont{Arial}{\textbf{Target\ Organisme}}}}} & \multicolumn{1}{!{\color[HTML]{000000}\vrule width 0pt}>{\raggedright}p{\dimexpr 2.39in+0\tabcolsep+0\arrayrulewidth}!{\color[HTML]{000000}\vrule width 0pt}}{\fontsize{11}{11}\selectfont{\textcolor[HTML]{000000}{\global\setmainfont{Arial}{\textbf{Rujukan}}}}} \\

\hhline{>{\arrayrulecolor[HTML]{666666}\global\arrayrulewidth=2pt}->{\arrayrulecolor[HTML]{666666}\global\arrayrulewidth=2pt}->{\arrayrulecolor[HTML]{666666}\global\arrayrulewidth=2pt}->{\arrayrulecolor[HTML]{666666}\global\arrayrulewidth=2pt}-}

\endfirsthead

\hhline{>{\arrayrulecolor[HTML]{666666}\global\arrayrulewidth=2pt}->{\arrayrulecolor[HTML]{666666}\global\arrayrulewidth=2pt}->{\arrayrulecolor[HTML]{666666}\global\arrayrulewidth=2pt}->{\arrayrulecolor[HTML]{666666}\global\arrayrulewidth=2pt}-}

\multicolumn{1}{!{\color[HTML]{000000}\vrule width 0pt}>{\raggedright}p{\dimexpr 1.23in+0\tabcolsep+0\arrayrulewidth}}{\fontsize{11}{11}\selectfont{\textcolor[HTML]{000000}{\global\setmainfont{Arial}{\textbf{Teknik}}}}} & \multicolumn{1}{!{\color[HTML]{000000}\vrule width 0pt}>{\raggedright}p{\dimexpr 2.18in+0\tabcolsep+0\arrayrulewidth}}{\fontsize{11}{11}\selectfont{\textcolor[HTML]{000000}{\global\setmainfont{Arial}{\textbf{Metode}}}}} & \multicolumn{1}{!{\color[HTML]{000000}\vrule width 0pt}>{\raggedright}p{\dimexpr 3.86in+0\tabcolsep+0\arrayrulewidth}}{\fontsize{11}{11}\selectfont{\textcolor[HTML]{000000}{\global\setmainfont{Arial}{\textbf{Target\ Organisme}}}}} & \multicolumn{1}{!{\color[HTML]{000000}\vrule width 0pt}>{\raggedright}p{\dimexpr 2.39in+0\tabcolsep+0\arrayrulewidth}!{\color[HTML]{000000}\vrule width 0pt}}{\fontsize{11}{11}\selectfont{\textcolor[HTML]{000000}{\global\setmainfont{Arial}{\textbf{Rujukan}}}}} \\

\hhline{>{\arrayrulecolor[HTML]{666666}\global\arrayrulewidth=2pt}->{\arrayrulecolor[HTML]{666666}\global\arrayrulewidth=2pt}->{\arrayrulecolor[HTML]{666666}\global\arrayrulewidth=2pt}->{\arrayrulecolor[HTML]{666666}\global\arrayrulewidth=2pt}-}\endhead



\multicolumn{1}{!{\color[HTML]{000000}\vrule width 0pt}>{\raggedright}p{\dimexpr 1.23in+0\tabcolsep+0\arrayrulewidth}}{} & \multicolumn{1}{!{\color[HTML]{000000}\vrule width 0pt}>{\raggedright}p{\dimexpr 2.18in+0\tabcolsep+0\arrayrulewidth}}{\fontsize{11}{11}\selectfont{\textcolor[HTML]{000000}{\global\setmainfont{Arial}{\textit{Quadrat\ plot}}}}} & \multicolumn{1}{!{\color[HTML]{000000}\vrule width 0pt}>{\raggedright}p{\dimexpr 3.86in+0\tabcolsep+0\arrayrulewidth}}{\fontsize{11}{11}\selectfont{\textcolor[HTML]{000000}{\global\setmainfont{Arial}{Tumbuhan}}}} & \multicolumn{1}{!{\color[HTML]{000000}\vrule width 0pt}>{\raggedright}p{\dimexpr 2.39in+0\tabcolsep+0\arrayrulewidth}!{\color[HTML]{000000}\vrule width 0pt}}{\fontsize{11}{11}\selectfont{\textcolor[HTML]{000000}{\global\setmainfont{Arial}{(Elzinga\ et\ al.,\ 2009)}}}} \\

\hhline{~>{\arrayrulecolor[HTML]{666666}\global\arrayrulewidth=0.5pt}->{\arrayrulecolor[HTML]{666666}\global\arrayrulewidth=0.5pt}->{\arrayrulecolor[HTML]{666666}\global\arrayrulewidth=0.5pt}-}



\multicolumn{1}{!{\color[HTML]{000000}\vrule width 0pt}>{\raggedright}p{\dimexpr 1.23in+0\tabcolsep+0\arrayrulewidth}}{} & \multicolumn{1}{!{\color[HTML]{000000}\vrule width 0pt}>{\raggedright}p{\dimexpr 2.18in+0\tabcolsep+0\arrayrulewidth}}{\fontsize{11}{11}\selectfont{\textcolor[HTML]{000000}{\global\setmainfont{Arial}{\textit{Point\ count}}}}} & \multicolumn{1}{!{\color[HTML]{000000}\vrule width 0pt}>{\raggedright}p{\dimexpr 3.86in+0\tabcolsep+0\arrayrulewidth}}{\fontsize{11}{11}\selectfont{\textcolor[HTML]{000000}{\global\setmainfont{Arial}{Burung}}}} & \multicolumn{1}{!{\color[HTML]{000000}\vrule width 0pt}>{\raggedright}p{\dimexpr 2.39in+0\tabcolsep+0\arrayrulewidth}!{\color[HTML]{000000}\vrule width 0pt}}{\fontsize{11}{11}\selectfont{\textcolor[HTML]{000000}{\global\setmainfont{Arial}{(Buckland,\ 2006)}}}} \\

\hhline{~>{\arrayrulecolor[HTML]{666666}\global\arrayrulewidth=0.5pt}->{\arrayrulecolor[HTML]{666666}\global\arrayrulewidth=0.5pt}->{\arrayrulecolor[HTML]{666666}\global\arrayrulewidth=0.5pt}-}



\multicolumn{1}{!{\color[HTML]{000000}\vrule width 0pt}>{\raggedright}p{\dimexpr 1.23in+0\tabcolsep+0\arrayrulewidth}}{} & \multicolumn{1}{!{\color[HTML]{000000}\vrule width 0pt}>{\raggedright}p{\dimexpr 2.18in+0\tabcolsep+0\arrayrulewidth}}{\fontsize{11}{11}\selectfont{\textcolor[HTML]{000000}{\global\setmainfont{Arial}{\textit{Line\ transect}}}}} & \multicolumn{1}{!{\color[HTML]{000000}\vrule width 0pt}>{\raggedright}p{\dimexpr 3.86in+0\tabcolsep+0\arrayrulewidth}}{\fontsize{11}{11}\selectfont{\textcolor[HTML]{000000}{\global\setmainfont{Arial}{Burung,\ mamalia\ dan\ primata}}}} & \multicolumn{1}{!{\color[HTML]{000000}\vrule width 0pt}>{\raggedright}p{\dimexpr 2.39in+0\tabcolsep+0\arrayrulewidth}!{\color[HTML]{000000}\vrule width 0pt}}{\fontsize{11}{11}\selectfont{\textcolor[HTML]{000000}{\global\setmainfont{Arial}{(Anderson\ et\ al.,\ 1979)}}}} \\

\hhline{~>{\arrayrulecolor[HTML]{666666}\global\arrayrulewidth=0.5pt}->{\arrayrulecolor[HTML]{666666}\global\arrayrulewidth=0.5pt}->{\arrayrulecolor[HTML]{666666}\global\arrayrulewidth=0.5pt}-}



\multicolumn{1}{!{\color[HTML]{000000}\vrule width 0pt}>{\raggedright}p{\dimexpr 1.23in+0\tabcolsep+0\arrayrulewidth}}{} & \multicolumn{1}{!{\color[HTML]{000000}\vrule width 0pt}>{\raggedright}p{\dimexpr 2.18in+0\tabcolsep+0\arrayrulewidth}}{\fontsize{11}{11}\selectfont{\textcolor[HTML]{000000}{\global\setmainfont{Arial}{\textit{Recce\ walk}}}}} & \multicolumn{1}{!{\color[HTML]{000000}\vrule width 0pt}>{\raggedright}p{\dimexpr 3.86in+0\tabcolsep+0\arrayrulewidth}}{\fontsize{11}{11}\selectfont{\textcolor[HTML]{000000}{\global\setmainfont{Arial}{Mamalia,\ reptil\ dan\ amfibi}}}} & \multicolumn{1}{!{\color[HTML]{000000}\vrule width 0pt}>{\raggedright}p{\dimexpr 2.39in+0\tabcolsep+0\arrayrulewidth}!{\color[HTML]{000000}\vrule width 0pt}}{} \\

\hhline{~>{\arrayrulecolor[HTML]{666666}\global\arrayrulewidth=0.5pt}->{\arrayrulecolor[HTML]{666666}\global\arrayrulewidth=0.5pt}-~}



\multicolumn{1}{!{\color[HTML]{000000}\vrule width 0pt}>{\raggedright}p{\dimexpr 1.23in+0\tabcolsep+0\arrayrulewidth}}{\multirow[c]{-5}{*}{\parbox{1.23in}{\fontsize{11}{11}\selectfont{\textcolor[HTML]{000000}{\global\setmainfont{Arial}{Observatif}}}}}} & \multicolumn{1}{!{\color[HTML]{000000}\vrule width 0pt}>{\raggedright}p{\dimexpr 2.18in+0\tabcolsep+0\arrayrulewidth}}{\fontsize{11}{11}\selectfont{\textcolor[HTML]{000000}{\global\setmainfont{Arial}{\textit{Visual\ encounter\ survey}}}}} & \multicolumn{1}{!{\color[HTML]{000000}\vrule width 0pt}>{\raggedright}p{\dimexpr 3.86in+0\tabcolsep+0\arrayrulewidth}}{\fontsize{11}{11}\selectfont{\textcolor[HTML]{000000}{\global\setmainfont{Arial}{Reptil\ dan\ amfibi}}}} & \multicolumn{1}{!{\color[HTML]{000000}\vrule width 0pt}>{\raggedright}p{\dimexpr 2.39in+0\tabcolsep+0\arrayrulewidth}!{\color[HTML]{000000}\vrule width 0pt}}{\multirow[c]{-2}{*}{\parbox{2.39in}{\fontsize{11}{11}\selectfont{\textcolor[HTML]{000000}{\global\setmainfont{Arial}{(Heyer\ et\ al.,\ 1994)}}}}}} \\

\hhline{>{\arrayrulecolor[HTML]{666666}\global\arrayrulewidth=0.5pt}->{\arrayrulecolor[HTML]{666666}\global\arrayrulewidth=0.5pt}->{\arrayrulecolor[HTML]{666666}\global\arrayrulewidth=0.5pt}->{\arrayrulecolor[HTML]{666666}\global\arrayrulewidth=0.5pt}-}



\multicolumn{1}{!{\color[HTML]{000000}\vrule width 0pt}>{\raggedright}p{\dimexpr 1.23in+0\tabcolsep+0\arrayrulewidth}}{} & \multicolumn{1}{!{\color[HTML]{000000}\vrule width 0pt}>{\raggedright}p{\dimexpr 2.18in+0\tabcolsep+0\arrayrulewidth}}{\fontsize{11}{11}\selectfont{\textcolor[HTML]{000000}{\global\setmainfont{Arial}{\textit{Passive\ acoustic\ monitoring}}}}} & \multicolumn{1}{!{\color[HTML]{000000}\vrule width 0pt}>{\raggedright}p{\dimexpr 3.86in+0\tabcolsep+0\arrayrulewidth}}{\fontsize{11}{11}\selectfont{\textcolor[HTML]{000000}{\global\setmainfont{Arial}{Burung,\ kelelawar\ pemakan\ serangga,\ amfibi}}}} & \multicolumn{1}{!{\color[HTML]{000000}\vrule width 0pt}>{\raggedright}p{\dimexpr 2.39in+0\tabcolsep+0\arrayrulewidth}!{\color[HTML]{000000}\vrule width 0pt}}{\fontsize{11}{11}\selectfont{\textcolor[HTML]{000000}{\global\setmainfont{Arial}{(Browning\ et\ al.,\ 2017)}}}} \\

\hhline{~>{\arrayrulecolor[HTML]{666666}\global\arrayrulewidth=0.5pt}->{\arrayrulecolor[HTML]{666666}\global\arrayrulewidth=0.5pt}->{\arrayrulecolor[HTML]{666666}\global\arrayrulewidth=0.5pt}-}



\multicolumn{1}{!{\color[HTML]{000000}\vrule width 0pt}>{\raggedright}p{\dimexpr 1.23in+0\tabcolsep+0\arrayrulewidth}}{} & \multicolumn{1}{!{\color[HTML]{000000}\vrule width 0pt}>{\raggedright}p{\dimexpr 2.18in+0\tabcolsep+0\arrayrulewidth}}{\fontsize{11}{11}\selectfont{\textcolor[HTML]{000000}{\global\setmainfont{Arial}{\textit{Live\ trap}}}}} & \multicolumn{1}{!{\color[HTML]{000000}\vrule width 0pt}>{\raggedright}p{\dimexpr 3.86in+0\tabcolsep+0\arrayrulewidth}}{\fontsize{11}{11}\selectfont{\textcolor[HTML]{000000}{\global\setmainfont{Arial}{Mamalia\ (Tikus\ dan\ bajing)}}}} & \multicolumn{1}{!{\color[HTML]{000000}\vrule width 0pt}>{\raggedright}p{\dimexpr 2.39in+0\tabcolsep+0\arrayrulewidth}!{\color[HTML]{000000}\vrule width 0pt}}{\fontsize{11}{11}\selectfont{\textcolor[HTML]{000000}{\global\setmainfont{Arial}{(Bovendorp\ et\ al.,\ 2017)}}}} \\

\hhline{~>{\arrayrulecolor[HTML]{666666}\global\arrayrulewidth=0.5pt}->{\arrayrulecolor[HTML]{666666}\global\arrayrulewidth=0.5pt}->{\arrayrulecolor[HTML]{666666}\global\arrayrulewidth=0.5pt}-}



\multicolumn{1}{!{\color[HTML]{000000}\vrule width 0pt}>{\raggedright}p{\dimexpr 1.23in+0\tabcolsep+0\arrayrulewidth}}{} & \multicolumn{1}{!{\color[HTML]{000000}\vrule width 0pt}>{\raggedright}p{\dimexpr 2.18in+0\tabcolsep+0\arrayrulewidth}}{\fontsize{11}{11}\selectfont{\textcolor[HTML]{000000}{\global\setmainfont{Arial}{\textit{Mist\ net}}}}} & \multicolumn{1}{!{\color[HTML]{000000}\vrule width 0pt}>{\raggedright}p{\dimexpr 3.86in+0\tabcolsep+0\arrayrulewidth}}{\fontsize{11}{11}\selectfont{\textcolor[HTML]{000000}{\global\setmainfont{Arial}{Burung,\ kelelawar\ pemakan\ buah}}}} & \multicolumn{1}{!{\color[HTML]{000000}\vrule width 0pt}>{\raggedright}p{\dimexpr 2.39in+0\tabcolsep+0\arrayrulewidth}!{\color[HTML]{000000}\vrule width 0pt}}{} \\

\hhline{~>{\arrayrulecolor[HTML]{666666}\global\arrayrulewidth=0.5pt}->{\arrayrulecolor[HTML]{666666}\global\arrayrulewidth=0.5pt}-~}



\multicolumn{1}{!{\color[HTML]{000000}\vrule width 0pt}>{\raggedright}p{\dimexpr 1.23in+0\tabcolsep+0\arrayrulewidth}}{} & \multicolumn{1}{!{\color[HTML]{000000}\vrule width 0pt}>{\raggedright}p{\dimexpr 2.18in+0\tabcolsep+0\arrayrulewidth}}{\fontsize{11}{11}\selectfont{\textcolor[HTML]{000000}{\global\setmainfont{Arial}{\textit{Harp\ trap}}}}} & \multicolumn{1}{!{\color[HTML]{000000}\vrule width 0pt}>{\raggedright}p{\dimexpr 3.86in+0\tabcolsep+0\arrayrulewidth}}{\fontsize{11}{11}\selectfont{\textcolor[HTML]{000000}{\global\setmainfont{Arial}{kelelawar}}}} & \multicolumn{1}{!{\color[HTML]{000000}\vrule width 0pt}>{\raggedright}p{\dimexpr 2.39in+0\tabcolsep+0\arrayrulewidth}!{\color[HTML]{000000}\vrule width 0pt}}{\multirow[c]{-2}{*}{\parbox{2.39in}{\fontsize{11}{11}\selectfont{\textcolor[HTML]{000000}{\global\setmainfont{Arial}{(Prasetyo\ et\ al.,\ 2011)}}}}}} \\

\hhline{~>{\arrayrulecolor[HTML]{666666}\global\arrayrulewidth=0.5pt}->{\arrayrulecolor[HTML]{666666}\global\arrayrulewidth=0.5pt}->{\arrayrulecolor[HTML]{666666}\global\arrayrulewidth=0.5pt}-}



\multicolumn{1}{!{\color[HTML]{000000}\vrule width 0pt}>{\raggedright}p{\dimexpr 1.23in+0\tabcolsep+0\arrayrulewidth}}{\multirow[c]{-5}{*}{\parbox{1.23in}{\fontsize{11}{11}\selectfont{\textcolor[HTML]{000000}{\global\setmainfont{Arial}{Penangkapan}}}}}} & \multicolumn{1}{!{\color[HTML]{000000}\vrule width 0pt}>{\raggedright}p{\dimexpr 2.18in+0\tabcolsep+0\arrayrulewidth}}{\fontsize{11}{11}\selectfont{\textcolor[HTML]{000000}{\global\setmainfont{Arial}{\textit{Camera\ trap}}}}} & \multicolumn{1}{!{\color[HTML]{000000}\vrule width 0pt}>{\raggedright}p{\dimexpr 3.86in+0\tabcolsep+0\arrayrulewidth}}{\fontsize{11}{11}\selectfont{\textcolor[HTML]{000000}{\global\setmainfont{Arial}{Burung\ terestrial,\ mamalia\ berukuran\ sedang\ -\ besar}}}} & \multicolumn{1}{!{\color[HTML]{000000}\vrule width 0pt}>{\raggedright}p{\dimexpr 2.39in+0\tabcolsep+0\arrayrulewidth}!{\color[HTML]{000000}\vrule width 0pt}}{\fontsize{11}{11}\selectfont{\textcolor[HTML]{000000}{\global\setmainfont{Arial}{(Rovero\ \&\ Zimmermann,\ 2016)}}}} \\

\hhline{>{\arrayrulecolor[HTML]{666666}\global\arrayrulewidth=2pt}->{\arrayrulecolor[HTML]{666666}\global\arrayrulewidth=2pt}->{\arrayrulecolor[HTML]{666666}\global\arrayrulewidth=2pt}->{\arrayrulecolor[HTML]{666666}\global\arrayrulewidth=2pt}-}



\end{longtable}

Setiap metode yang digunakan dalam tabel \ref{tab:tab1} membutuhkan \emph{Standard sampling unit} (SSU) / unit cuplik, yaitu suatu kelompok pengamatan atau data yang dikumpulkan secara independen berdasarkan metode yang terdefinisi dengan jelas dan dilakukan berulang-ulang. SSU biasanya menggambarkan jumlah sampel (N) dalam tes statistik. Unit cuplik harus konsisten sepanjang studi berlangsung. Sebagai contoh; jika unit cuplik yang digunakan adalah plot sirkular untuk pengamatan burung dengan radius pengamatan 50 meter, maka plot lainnya harus dengan radius yang sama. Dalam melakukan pencuplikan (sampling), asumsi dasar yang harus dipenuhi adalah cuplikan itu mampu menggambarkan keseluruhan area yang disurvei. Oleh karena itu, jumlah sampel yang diambil harus cukup banyak untuk dapat dijadikan perwakilan area.

Salah satu cara yang dapat digunakan untuk memaksimalkan ukuran sampel adalah pengulangan / replikasi. Replikasi pada dasarnya adalah melakukan pengambilan secara berulang atau beberapa kali di dalam \emph{SSU}. Replikasi mengurangi bias dalam pengamatan, sebagai contoh; Jika pengamatan dilakukan hanya 1 kali di suatu plot, besar kemungkinan akan banyak spesies yang luput dari pengamatan, namun jika pengamatan dilakukan selama beberapa hari, besar kemungkinan akan banyak temuan spesies baru dibandingkan hari sebelumnya.

Seringkali analisa statistik dalam kajian satwa liar juga membutuhkan sejumlah data yang cukup banyak. Sebagai contoh; Jika kita memasang 1 kamera jebak selama 1 malam untuk menangkap gambar rusa (dalam statistik ukuran sampel tersebut adalah N=1), dan pada malam itu tidak tertangkap rusa sama sekali, belum tentu dikawasan tersebut tidak terdapat rusa (\emph{false-absence}), sebaliknya jika 10 unit kamera jebak dipasang selama 1 malam (N=10) dan 5 diantaranya menangkap gambar rusa, kita juga belum bisa mengambil kesimpulan, bahwa hanya 5 area yang dihuni oleh rusa. Jika 30 kamera jebak dipasang pada kawasan hutan dan masing-masing aktif selama 30 hari (N=900), maka kita mungkin baru bisa mengambil kesimpulan yang lebih akurat dan presisi. Penggunaan statistik dapat memberikan tingkat keyakinan terhadap keputusan yang kita ambil berdasarkan jumlah unit cuplik yang digunakan.

\hypertarget{penentuan-peletakan-lokasi-unit-cuplik}{%
\section*{Penentuan peletakan lokasi unit cuplik}\label{penentuan-peletakan-lokasi-unit-cuplik}}
\addcontentsline{toc}{section}{Penentuan peletakan lokasi unit cuplik}

Rancangan peletakan lokasi unit cuplik harus dapat menggambarkan keseluruhan lokasi secara umum, oleh karena itu harus disesuaikan dengan tujuan studi dan kondisi habitat. Metode-metode penentuan lokasi tersebut diantaranya adalah \emph{purposive, random, dan stratified}.

\hypertarget{purposive-sampling}{%
\subsection*{\texorpdfstring{\emph{Purposive sampling}}{Purposive sampling}}\label{purposive-sampling}}
\addcontentsline{toc}{subsection}{\emph{Purposive sampling}}

Metode ini dikenal juga dengan \emph{judgemental selective}, artinya penempatan unit cuplik dipilih secara langsung berdasarkan subjektivitas peneliti dengan pertimbangan tertentu, misalnya area dimana keberadaan jenis telah diketahui. Metode ini umumnya bersifat kualitatif dan tidak melibatkan pengolahan statistik karena setiap titik cupliknya diperlakukan berbeda, atau tidak semua titik di dalam populasi cuplik memiliki peluang untuk dipilih. Metode ini baik diaplikasikan untuk pendataan jenis dimana lokasi keberadaan jenis telah diketahui atau diperkirakan. Contoh dari purposive sampling ini diantaranya pemilihan sungai untuk survey katak. Sungai dipilih dengan pertimbangan bahwa sungai sebagai habitat katak akan memberikan probabilitas perjumpaan yang lebih besar dibanding kawasan lainnya.

\hypertarget{random-sampling}{%
\subsection*{\texorpdfstring{\emph{Random sampling }}{Random sampling }}\label{random-sampling}}
\addcontentsline{toc}{subsection}{\emph{Random sampling }}

Pada metode ini, penempatan unit cuplik ditetapkan secara acak di dalam batas-batas area yang telah ditetapkan sebelumnya. Asumsi yang digunakan pada metode ini adalah bahwa setiap titik memiliki probabilitas yang sama untuk dipilih, sehingga semakin banyak titik yang dicuplik akan semakin mewakili populasi area. Metode ini baik diaplikasikan pada area survey dimana sebaran populasi cenderung normal dan memiliki kondisi habitat yang realtif homogen, contohnya survey burung di area hutan dengan tipe habitat yang seragam.

\begin{figure}

{\centering \includegraphics[width=1\linewidth]{images/rsamp} 

}

\caption{Ilustrasi peletakan unit cuplik secara acak (random) [kiri] titik hitung [kanan] transek}\label{fig:sandomsampling}
\end{figure}

\hypertarget{stratified-sampling}{%
\subsection*{\texorpdfstring{\emph{Stratified sampling}}{Stratified sampling}}\label{stratified-sampling}}
\addcontentsline{toc}{subsection}{\emph{Stratified sampling}}

Stratifikasi umumnya digunakan untuk membatasi sebaran populasi cuplik dimana terdapat perbedaan tipe habitat di dalam area survei. Asumsi yang diambil adalah bahwa tipe habitat yang berbeda cenderung memberikan keterwakilan populasi yang berbeda secara signifikan.

Langkah pembagian strata paling mudah adalah dengan melihat kategori tutupan lahan atau tipe habitat sehingga akan didapat sejumlah strata sebanyak tipe habitat yang ada. Di masing-masing strata kemudian dilakukan pemilihan unit cuplik secara acak. Dengan demikian, setiap strata memiliki keterwakilan cuplik yang setara

Meski demikian, jumlah pembagian unit cuplik disetiap strata bergantung pada prinsip variabilitas spesies. Sebagai contoh: {[}i{]} Hutan sekunder alami cenderung memiliki kekayaan jenis lebih tinggi dibandingkan dengan perkebunan, sehingga jumlah unit cuplik dikawasan hutan harus lebih banyak dibandingkan di perkebunan. {[}ii{]} Sebaliknya, apabila kita membandingkan dua strata yang belum diketahui variabilitasnya seperti hutan sekunder dataran rendah dan hutan sekunder dataran tinggi maka jumlah unit cupliknya diupayakan sama.

\begin{figure}

{\centering \includegraphics[width=1\linewidth]{images/randomstrat} 

}

\caption{Ilustrasi peletakan unit cuplik menggunakan strata (stratified)}\label{fig:randomstrat}
\end{figure}

Konsep stratifikasi juga bisa dilakukan paska survei, sebagai contoh jika transek atau plot yang digunakan sama antara vegetasi dan fauna. Kemudian, hasil survei vegetasi menemukan bahwa terdapat area dengan kepadatan biomasa yang tinggi, rendah dan sedang atau memiliki area dengan kelompok tipe vegetasi yang berbeda secara signifikan. Maka analisa untuk fauna bisa dikelompokan sesuai dengan strata tersebut.

\hypertarget{bentuk-dan-jumlah-unit-cuplik}{%
\section*{Bentuk dan jumlah unit cuplik}\label{bentuk-dan-jumlah-unit-cuplik}}
\addcontentsline{toc}{section}{Bentuk dan jumlah unit cuplik}

Unit cuplik yang sering dipergunakan oleh FFI`s IP untuk kajian biodiversitas biasanya adalah transek sepanjang 2 km untuk mamalia dan herpetofauna dengan kombinasi kuadrat plot untuk tumbuhan dan plot sirkular untuk burung disepanjang transek tersebut. Satu unit cuplik dengan desain tersebut dapat mencakup 2 ha untuk mamalia dan herpetofauna, 1,25 ha untuk vegetasi, serta 4.8 ha untuk burung.

Jumlah unit cuplik disesuaikan dengan ketersediaan sumber daya yang ada. Dalam survei biodiversitas, tidak ada jumlah minimum yang pasti, namun umumnya peneliti mengevaluasi kecukukupan jumlah unit cuplik dari kurva akumulasi spesies paska survei selesai. Oleh karena itu penting untuk melihat referensi pada penelitian lain untuk menentukan jumlah unit cuplik yang sesuai pada karakteristik habitat atau lokasi yang sama sebelum survei dimulai.

Pendekatan lain yang dapat digunakan untuk menentukan jumlah unit cuplik minimum adalah menggunakan metode yang dikembangkan oleh \href{https://winrock.org/document/winrock-sample-plot-calculator-spreadsheet-tool/}{Winrock International}, dengan memasukan luas setiap strata, ukuran plot dan nilai estimasi rerata kandungan biomassa pada tipe habitat serupa. \emph{Tools} tersebut sering digunakan dan diakui untuk kajian dan monitoring proyek-proyek restorasi.

\hypertarget{protokol-survei}{%
\chapter*{Protokol Survei}\label{protokol-survei}}
\addcontentsline{toc}{chapter}{Protokol Survei}

\hypertarget{avifauna}{%
\section*{Avifauna}\label{avifauna}}
\addcontentsline{toc}{section}{Avifauna}

Pengamatan burung atau avifauna yang biasa dilakukan oleh FFI`s IP mengadopsi dua metode utama yaitu metode titik hitung di transek (\emph{Point transect}) (Buckland, 2006) dan daftar jenis MacKinnon (\emph{Mackinnon lists}) (MacKinnon \& Phillipps, 1993). Pada dasarnya metode point transect merupakan modifikasi dari metode titik hitung, namun unit sampelnya berada dalam transek yang sudah ditetapkan, metode ini efektif digunakan pada hutan tropis, dimana jalurnya seringkali sulit untuk dilalui dan burung menghuni seluruh strata hutan dari permukaan tanah hingga diatas tajuk. Dengan fokus pada titik tertentu di dalam transek, deteksi burung jadi lebih efektif. Pada Mackinnon lists survei dilakukan bisa di jalur transek atau pun di luar transek. Kedua metode ini saling melengkapi dalam pengumpulan data jenis-jenis burung

\hypertarget{persiapan-tim}{%
\subsection*{Persiapan Tim}\label{persiapan-tim}}
\addcontentsline{toc}{subsection}{Persiapan Tim}

Tim avifauna idealnya terdiri dari 2 orang, yaitu pengamat utama dan asisten lapangan. Dengan peran dan tanggung jawab yang terangkum dalam tabel \ref{tab:tbta}.

\begin{longtable}[]{@{}
  >{\raggedright\arraybackslash}p{(\columnwidth - 4\tabcolsep) * \real{0.1765}}
  >{\raggedright\arraybackslash}p{(\columnwidth - 4\tabcolsep) * \real{0.4118}}
  >{\raggedright\arraybackslash}p{(\columnwidth - 4\tabcolsep) * \real{0.4118}}@{}}
\caption{\label{tab:tbta} Peran dan tanggung jawab tim avifauna}\tabularnewline
\toprule()
\begin{minipage}[b]{\linewidth}\raggedright
Peran
\end{minipage} & \begin{minipage}[b]{\linewidth}\raggedright
Tanggung jawab
\end{minipage} & \begin{minipage}[b]{\linewidth}\raggedright
Syarat khusus
\end{minipage} \\
\midrule()
\endfirsthead
\toprule()
\begin{minipage}[b]{\linewidth}\raggedright
Peran
\end{minipage} & \begin{minipage}[b]{\linewidth}\raggedright
Tanggung jawab
\end{minipage} & \begin{minipage}[b]{\linewidth}\raggedright
Syarat khusus
\end{minipage} \\
\midrule()
\endhead
Pengamat utama & Mengamati dan mengidentifikasi burung pada lokasi yang disurvei, kemudian memberikan informasi pada pencatat mengenai data yang dibutuhkan seperti yang tertera pada lembar data & Memahami protokol serta identifikasi jenis burung dan penggunaan peralatan pendukung survei \\
Asisten lapangan & Mencatat data temuan survei dan juga sebagai pencatat waktu (\emph{time keeper}) & Memahami protokol survei avifauna dengan baik \\
\bottomrule()
\end{longtable}

\hypertarget{peralatan}{%
\subsection*{Peralatan}\label{peralatan}}
\addcontentsline{toc}{subsection}{Peralatan}

\begin{longtable}[]{@{}
  >{\raggedright\arraybackslash}p{(\columnwidth - 4\tabcolsep) * \real{0.1765}}
  >{\raggedright\arraybackslash}p{(\columnwidth - 4\tabcolsep) * \real{0.4118}}
  >{\raggedright\arraybackslash}p{(\columnwidth - 4\tabcolsep) * \real{0.4118}}@{}}
\caption{\label{tab:tbpa} Peralatan yang dibutuhkan tim avifauna}\tabularnewline
\toprule()
\begin{minipage}[b]{\linewidth}\raggedright
Peralatan
\end{minipage} & \begin{minipage}[b]{\linewidth}\raggedright
Penggunaan
\end{minipage} & \begin{minipage}[b]{\linewidth}\raggedright
Spesifikasi
\end{minipage} \\
\midrule()
\endfirsthead
\toprule()
\begin{minipage}[b]{\linewidth}\raggedright
Peralatan
\end{minipage} & \begin{minipage}[b]{\linewidth}\raggedright
Penggunaan
\end{minipage} & \begin{minipage}[b]{\linewidth}\raggedright
Spesifikasi
\end{minipage} \\
\midrule()
\endhead
Alat Tulis & Pencatatan data dan penandaan & Kuat, tidak mudah luntur \\
Lembar data & Lembar pencatatan data & Tahan air \\
Alat Navigasi (GPS, Peta dan Kompas) & Untuk navigasi sekaligus penanda lokasi geografis & Tahan air \\
Binokuler & Untuk melihat dan mengidentifikasi burung & Perbesaran lensa minimal 8 x 40 atau 7 x 50 \\
Kamera & Untuk dokumentasi burung dan identifikasi lebih lanjut & DSLR dengan lensa tele 300 -- 400 mm. Alternatif lainnya dapat menggunakan kamera digital \emph{prosummer} dengan perbesaran optik diatas 30x \\
Perekam suara genggam & Merekam suara burung untuk identifikasi lebih lanjut & Perekam suara digital dengan fitur \emph{directional microphone} \\
Perekam suara pasif & Merekam suara burung untuk identifikasi burung yang sensitif & Tahan air. Perangkat yang biasa digunkan adalah \emph{audiomoth} (Hill et al., 2019) \\
\bottomrule()
\end{longtable}

\hypertarget{protokol-pengamatan}{%
\subsection*{Protokol Pengamatan}\label{protokol-pengamatan}}
\addcontentsline{toc}{subsection}{Protokol Pengamatan}

\hypertarget{titik-hitung}{%
\subsubsection*{Titik hitung}\label{titik-hitung}}
\addcontentsline{toc}{subsubsection}{Titik hitung}

Protokol untuk survei dengan metode titik hitung dalam transek yang dilakukan oleh FFI`s IP menggunakan enam buah titik hitung dengan rentang antar titik berjarak 200m sehingga akan membentuk garis transek sejauh 1 Km (Gambar \ref{fig:figpc}). Radius pengamatan per titik adalah 50m dari titik pusat. Titik pusat yang dimaksud adalah titik yang telah ditentukan. Pengamatan menggunakan titik hitung mengikuti asumsi-asumsi berikut ini:

\begin{enumerate}
\def\labelenumi{\arabic{enumi}.}
\tightlist
\item
  Burung tidak mendekati pengamat atau terbang;
\item
  Burung yang ada dalam titik cuplik dapat terdeteksi 100\%;
\item
  Burung tidak bergerak selama perhitungan;
\item
  Burung berperilaku bebas (tidak tergantung satu sama lain);
\item
  Pelanggaran terhadap asumsi tersebut tidak berpengaruh terhadap habitat atau desain studi;
\item
  Estimasi jarak akurat;
\item
  Burung dapat teridentifikasi dengan baik seluruhnya.
\end{enumerate}

Dalam pelaksanaannya, pengamat berhenti pada suatu titik pengamatan selama 20 menit untuk mengamati dan mencatat jenis burung yang dapat diidentifikasi di sekitar lokasi penelitian. Setelah 20 menit, pengamat kemudian berpindah ke titik pengamatan lain dan kemudian melakukan pengamatan lagi di titik pengamatan tersebut dengan waktu yang sama yaitu selama 20 menit. Jumlah titik pada setiap jalur adalah enam titik, dengan jarak masing-masing titik 200 m, sehingga panjang jalur pengamatannya adalah 1 km. Pengamatan dilakukan pada pagi hari pukul 06.00-09.00 WIB dan sore hari pukul 15.30-18.00 WIB. Perjumpaan terhadap jenis burung di luar titik pengamatan tidak diperhitungkan. Pada setiap jalur pengamatan dilakukan pengulangan pengamatan sebanyak dua kali. Pengamatan dilakukan melalui perjumpaan langsung dengan objek (visual) dan melalui suara. Parameter yang dicatat adalah jenis burung, jumlah yang ditemukan dan aktifitas. Jika memungkinkan, maka jarak setiap burung yang dijumpai terhadap pengamat juga diukur, dengan data seperti itu maka kepadatan burung juga dapat dihitung dengan konsep distance sampling (Buckland et al., 2015).

\begin{figure}

{\centering \includegraphics[width=1\linewidth]{images/pc_ilustration} 

}

\caption{Ilustrasi titik hitung di transek}\label{fig:figpc}
\end{figure}

\textbf{Cara Pelaksanaan :}
1. Sebelum menuju ke titik hitung, pengamat sudah menentukan lokasi titik -- titik hitung tersebut di GPS.
2. Pengamat menuju titik yang sudah ditentukan di dalam transek, dimana jarak antar point sepanjang 200 meter.
3. Setiap titik ditandai di dalam GPS
4. Pengamat berdiri di titik tengah dari point yang sudah ditentukan.
5. Pengamat mengamati dan mencatat burung yang terdengar ataupun terlihat selama 20 menit ke dalam lembar pengamatan titik hitung (Gambar \ref{fig:ldpth})
6. Untuk penggunaan perekam suara, bisa digunakan selama 20 menit pengamatan atau ketika mendengar suara-suara yang menarik.
7. Asisten dapat membantu mengukur parameter lingkungan disekitar lokasi pengamatan selama durasi pengamatan kedalam lembar data parameter lingkungan (Gambar \ref{fig:ldppl}) secara semi-kuantitatif.

\hypertarget{daftar-jenis-mackinnon}{%
\subsubsection*{Daftar jenis MacKinnon}\label{daftar-jenis-mackinnon}}
\addcontentsline{toc}{subsubsection}{Daftar jenis MacKinnon}

Metode ini pada dasarnya membuat sejumlah daftar yang berisi catatan nama jenis-jenis burung yang dijumpai untuk mendapat gambaran cepat mengenai kekayaan dan komposisi jenis burung pada suatu wilayah. Rincian prosedur penyusunan daftar dijelaskan di bawah ini.

\textbf{Cara Pelaksanaan:}
1. Berjalan di suatu habitat, seperti perjalanan dari desa menuju camp, di sekitar camp, dari camp menuju transek, transek satu kilo diluar point dan ketika perjalanan dari point menuju point yang lain dan mencatat semua jenis burung yang dijumpai sampai tercatat 20 jenis burung dalam satu daftar. Satu jenis burung hanya dicatat satu kali saja dalam satu daftar ini, meskipun dijumpai beberapa kali

\begin{enumerate}
\def\labelenumi{\arabic{enumi}.}
\setcounter{enumi}{1}
\item
  Setelah tercatat 20 jenis burung, lalu membuat daftar yang baru untuk mencatat jenis-jenis yang dijumpai selanjutnya (daftar no.2). Apabila dijumpai jenis yang pernah tercatat dalam daftar pertama maka tetap dicatat dalam daftar kedua, tetapi sebagaimana dalam pembuatan daftar pertama, jenis yang sudah dicatat dalam daftar kedua tidak boleh dicatat lagi meskipun dijumpai beberapa kali (di dalam satu daftar tidak boleh ada pengulangan jenis). Jika suatu spesies ditemukan kembali dalam 1 daftar yang belum mencapai 20 spesies, maka spesies tersebut hanya dihitung sebagai tambahan populasi pada spesies yang sama (bukan spesies baru)
\item
  Jika menemukan spesies yang menarik maka di tandai posisinya di dalam GPS, begitu juga jika mendengar suara yang menarik maka bisa di rekam di perekam suara.
\end{enumerate}

Metode ini meskipin sederhana, namun membutuhkan pengetahuan yang baik terhadap ekologi dan perilaku burung-burung di area survei. Terkadang pengamat boleh untuk duduk bersembunyi sebentar saat berada habitat yang sedang berbuah dan berbunga untuk melihat dan mendengar burung-burung yang berkunjung. Lampiran Gambar \ref{fig:ldpml}, merupakan contoh lembar data untuk metode daftar jenis MacKinnon.

\hypertarget{perekam-suara-pasif}{%
\subsubsection*{Perekam suara pasif}\label{perekam-suara-pasif}}
\addcontentsline{toc}{subsubsection}{Perekam suara pasif}

Untuk melengkapi daftar jenis burung-burung yang mungkin terlalu sensitif terhadap keberadaan manusia / pengamat, maka penggunaan perekam suara dapat dijadikan alternatif karena mampu merekam tanpa kehadiran pengamat selama waktu yang dibutuhkan dan tidak akan ada bias dalam identifikasi karena memiliki data suara yang terdokumentasikan dengan baik. Dalam praktiknya, FFI`s IP seringkali menggunakan perangkat perekam suara \emph{audiomoth} untuk merekam suara burung-burung di hutan.

Perekam suara dapat ditempatkan disetiap titik hitung sebagai data pelengkap atau lokasi spesifik lainnya yang diperkirakan memiliki kelimpahan burung dengan jarak minimal antar perekam suara 250 - 1000 meter. Setiap perekam suara diaktifkan minimal 1 x 24 jam agar burung diurnal dan nokturnal dapat terekam. Prinsipnya semakin lama di aktifkan maka data yang diperoleh semakin baik, perangkat ini dapat diaktifkan hingga sekitar 10 hari dengan baterai tipe alkalin dengan pengaturan 5 menit merekam dan 30 menit jeda. Adapun protokol penggunaan perekam suara adalah sebagai berikut;

\begin{enumerate}
\def\labelenumi{\arabic{enumi}.}
\item
  Melakukan pengaturan perangkat dengan spesifikasi sebagai berikut

  \begin{itemize}
  \tightlist
  \item
    Sample rate; 48 Khz
  \item
    Gain; Medium
  \item
    Sleep duration; 1800s
  \item
    Recording duration; 300s
  \end{itemize}
\item
  Pastikan pengaturan sudah sesuai dengan yang kita inginkan, dengan melakukan simulasi terlebih dahulu
\item
  Beri label pada setiap perangkat untuk membedakan antar perekam suara
\item
  Bungkus perangkat dengan plastik atau penutup kedap air dan pasang pada batang pohon dengan ketinggian sekitar 2 meter.
\item
  Catat kordinat pemasangan, waktu mulai dan waktu berakhirnya pada lembar pengamatan
\end{enumerate}

\hypertarget{herpetofauna}{%
\section*{Herpetofauna}\label{herpetofauna}}
\addcontentsline{toc}{section}{Herpetofauna}

Survei herpetofauna biasanya didasarkan pada tujuan untuk mendapatkan informasi mengenai:

\begin{enumerate}
\def\labelenumi{\arabic{enumi}.}
\tightlist
\item
  Daftar jenis herpetofauna pada suatu lokasi
\item
  Densitas atau kelimpahan relatif pada suatu lokasi
\item
  Penyebaran jenis
\item
  Aspek spesifik pada satu jenis seperti penggunaan habitat, pola aktifitas, biologi, reproduksi, dan sebagainya
\end{enumerate}

Perbedaan tujuan survei berpengaruh terhadap metode yang digunakan, usaha (effort), biaya dan waktu. Untuk survei yang bertujuan memperoleh daftar jenis, untuk memperoleh sebanyak mungkin jenis pada suatu lokasi perlu dilakukan pencarian aktif dan jebakan (pasif) dengan waktu yang mencakup siang dan malam hari. Hasil survei juga tergantung pada kondisi cuaca dan musim. Terdapat beberapa kendala dalam survei daftar jenis, diantaranya lama pengamatan yang mempengaruhi hasil temuan, biaya dan efektifitas kerja. Perbandingan hasil survei baik pada lokasi yang sama ataupun berbeda, perlu memperhatikan usaha (effort), musim dilakukan survei, dan metode yang digunakan

Metode yang biasa digunakan oleh FFI`s IP adalah pencarian langsung dengan metode Survei Perjumpaan Visual (Visual Encounter surveis - VES) (Gambar \ref{fig:figves}). Metode ini dilakukan dengan menyusuri area atau habitat herpetofauna dengan batasan waktu tertentu. Pencarian herpetofauna difokuskan pada tipe jenisnya, pencarian di tajuk pohon, batang dan ranting untuk jenis arboreal, kemudian pencarian di balik serasah, batu, kayu lapuk untuk jenis terestrial, terdapat juga jenis yang hidup di dalam tanah (fosorial) dan perairan seperti sungai, danau, dan kolam.

Metode VES memiliki batasan waktu pencarian di luar waktu penangkapan dan pencatatan. Penentuan batas waktu total biasanya 2 jam per orang per pengambilan sampel. Metode ini dapat dilakukan pada jalur transek, plot petak, aliran sungai, kolam dan lainya. Metode diterapkan dengan melakukan survei lokasi pengamatan dengan menentukan lokasi pengamatan yang memiliki dugaan habitat herpetofauna, membuat sedikit jalur untuk memudahkan pengamatan malam hari dan memberi tanda pada lokasi. Data yang dikumpulkan adalah ukuran lokasi, karakteristik habitat berupa vegetasi dominan, kerapatan tajuk, kondisi fisik lokasi dan dokumentasi lokasi.

\begin{figure}

{\centering \includegraphics[width=1\linewidth]{images/ves_ilustration} 

}

\caption{Desain VES dalam transek dengan panjang 1 km dan lebar kiri – kanan transek adalah 5 - 10 m.}\label{fig:figves}
\end{figure}

\textbf{Persiapan}
Langkah awal sebelum melakukan survei salah satunya adalah pengumpulan informasi terkait lokasi survei dan data dasar pada lokasi. Informasi lokasi survei yang dimaksud antara lain, status administrasi, letak dan aksesibilitas lokasi survei. Sedangkan informasi data dasar berupa peta lokasi, data hasil survei sebelumnya dan data pendukung lain yang diperlukan. Informasi tersebut dapat diperoleh secara langsung kepada pengelola lokasi survei atau melalu lembaga yang menyediakan informasi terkait.

\hypertarget{persiapan-tim-1}{%
\subsection*{Persiapan Tim}\label{persiapan-tim-1}}
\addcontentsline{toc}{subsection}{Persiapan Tim}

Tim herpetofauna idealnya terdiri dari 3 orang, yaitu seorang pencatat, pencari atau pengamat, dan asisten lapangan (Tabel \ref{tab:tabtmh}). Namun pada kondisi tertentu, survei dapat dilakukan oleh 2 orang saja dengan konsekuensi seorang pencatat berperan juga menjadi spotter pada saat yang sama.

\begin{longtable}[]{@{}
  >{\raggedright\arraybackslash}p{(\columnwidth - 4\tabcolsep) * \real{0.1765}}
  >{\raggedright\arraybackslash}p{(\columnwidth - 4\tabcolsep) * \real{0.4118}}
  >{\raggedright\arraybackslash}p{(\columnwidth - 4\tabcolsep) * \real{0.4118}}@{}}
\caption{\label{tab:tabtmh} Peran dan tanggung jawab tim herpetofauna}\tabularnewline
\toprule()
\begin{minipage}[b]{\linewidth}\raggedright
Peran
\end{minipage} & \begin{minipage}[b]{\linewidth}\raggedright
Tanggung jawab
\end{minipage} & \begin{minipage}[b]{\linewidth}\raggedright
Syarat khusus
\end{minipage} \\
\midrule()
\endfirsthead
\toprule()
\begin{minipage}[b]{\linewidth}\raggedright
Peran
\end{minipage} & \begin{minipage}[b]{\linewidth}\raggedright
Tanggung jawab
\end{minipage} & \begin{minipage}[b]{\linewidth}\raggedright
Syarat khusus
\end{minipage} \\
\midrule()
\endhead
Peneliti utama; Pengamat (\emph{Spotter}) & Mencari herpetofauna pada lokasi yang disurvei, mengambil foto dari individu herpetofauna, kemudian memberikan informasi pada pencatat mengenai data yang dibutuhkan seperti yang tertera pada lembar data & Memahami protokol serta identifikasi jenis dan penggunaan peralatan pendukung survei \\
Asisten I: Pencatat & Mencatat data temuan survei dan juga sebagai pencatat waktu (\emph{time keeper}) & Memahami protokol survei herpetofauna dengan baik \\
Asisten II: Penunjuk jalan & Porter atau penunjuk jalan saat survei dan membantu pencarian herpetofauna & Memahami area survei dan kemampuan membaca jalur \\
\bottomrule()
\end{longtable}

\hypertarget{peralatan-1}{%
\subsection*{Peralatan}\label{peralatan-1}}
\addcontentsline{toc}{subsection}{Peralatan}

Peralatan survei yang dipersiapkan disesuaikan dengan waktu, metode dan jumlah surveior. Menyusun daftar peralatan perlu pengelompokan sesuai fungsi untuk mempermudah persiapan, secara umum disusun sebagai berikut:

\begin{longtable}[]{@{}
  >{\raggedright\arraybackslash}p{(\columnwidth - 2\tabcolsep) * \real{0.3000}}
  >{\raggedright\arraybackslash}p{(\columnwidth - 2\tabcolsep) * \real{0.7000}}@{}}
\caption{\label{tab:tbph} Peralatan yang dibutuhkan tim herpetofauna}\tabularnewline
\toprule()
\begin{minipage}[b]{\linewidth}\raggedright
Kelompok
\end{minipage} & \begin{minipage}[b]{\linewidth}\raggedright
Alat
\end{minipage} \\
\midrule()
\endfirsthead
\toprule()
\begin{minipage}[b]{\linewidth}\raggedright
Kelompok
\end{minipage} & \begin{minipage}[b]{\linewidth}\raggedright
Alat
\end{minipage} \\
\midrule()
\endhead
Navigasi & GPS, kompas, dan peta \\
Survei dan koleksi data & Senter kepala (\emph{headlamp}), jam tangan, penggaris, meteran, plastik sampel, kantong kain, kamera, alat tulis, lembar data, Sepatu boot \\
Preservasi & Kotak specimen, kain kasa, syringe, alat bedah, kertas label, benang, gunting, alkohol, formalin, toples spesimen \\
\bottomrule()
\end{longtable}

\textbf{Baterai}
Alat-alat seperti \emph{headlamp}, GPS dan Kamera. Kalkulasikan kebutuhan baterai tersebut sebelum melakukan survei. Informasi lama daya tahan baterai saat digunakan oleh suatu alat sangat penting untuk menghitung berapa banyak jumlah baterai yang digunakan setiap harinya.

Tipe baterai yang digunakan perlu menjadi perhatian. Disarankan menggunakan baterai yang memiliki ketahanan yang lama seperti baterai alkaline \emph{non-rechargeable}. Hal tersebut berdasarkan pada pertimbangan biasanya tidak terdapat sumber listrik pada lokasi survei.

Selain peralatan survei, perlu dipersiapkan perlengkapan pribadi untuk kegiatan di lapangan, logistik bahan lapang, makanan dan obat-obatan. Kegiatan di lapangan sebaiknya disusun dengan jadwal kegiatan perhari dan tabel waktu selama survei agar efektif.

\hypertarget{protokol-pengamatan-1}{%
\subsection*{Protokol Pengamatan}\label{protokol-pengamatan-1}}
\addcontentsline{toc}{subsection}{Protokol Pengamatan}

\hypertarget{waktu-pengamatan}{%
\subsubsection*{Waktu Pengamatan}\label{waktu-pengamatan}}
\addcontentsline{toc}{subsubsection}{Waktu Pengamatan}

Pengamatan pada waktu siang hari dilakukan dengan estimasi waktu pukul 08.00 -- 10.00 untuk mencari jenis yang aktif pada siang hari (diurnal), terutama untuk kelompok reptil (kadal, ular). Pengamatan pada waktu malam hari dilakukan dengan estimasi waktu pukul 19.00 -- 22.00 pada lokasi yang telah disurvei saat siang hari. Tidak ada batasan berapa jumlah pengamat pada tiap pengamatan, namun demi keamanan minimal jumlah pengamat adalah 2 orang

\hypertarget{persiapan-pengamatan}{%
\subsubsection*{Persiapan Pengamatan}\label{persiapan-pengamatan}}
\addcontentsline{toc}{subsubsection}{Persiapan Pengamatan}

\begin{itemize}
\tightlist
\item
  Persiapkan alat yang diperlukan dan lakukan pengecekan kondisinya. Terutama GPS, \emph{headlamp} dan kamera serta persiapkan baterai cadangannya.
\item
  Cek kesehatan anggota tim
\item
  Lakukan pembagian tugas. Pembagian tugas yang dimaksud adalah pembagian kerja kepada tiap anggota tim berupa pencari / pengamat utama, dokumentasi, pengukur dan pencatat data.
\item
  Cek kembali kondisi alat -- alat sesampainya di lokasi observasi yang telah ditentukan dan sesaat sebelum melakukan pengamatan
\end{itemize}

\hypertarget{koleksi-data}{%
\subsubsection*{Koleksi Data}\label{koleksi-data}}
\addcontentsline{toc}{subsubsection}{Koleksi Data}

\begin{itemize}
\tightlist
\item
  Hal pertama yang dilakukan adalah pencatatan waktu, kondisi cuaca dan lingkungan (suhu dan kelembapan udara).
\item
  Mulai mencari dengan berjalan perlahan sambil menyorotkan lampu pada pada substrat tajuk, batang, serasah, lubang, di balik batu dan kayu mati. Indikasi adanya herpetofauna dapat dilihat dengan pantulan cahaya dari mata satwa tersebut. Terkadang dapat diketahui melalui suara, terutama untuk katak dan kodok.
\item
  Saat menemukan satwa, hal pertama yang dicatat adalah waktu ditemukan, tempat ditemukan (jarak horizontal dari badan air, jarak vertical dari permukaan tanah), substrat, aktifitas, dan nama jenis jika sudah dapat diketahui pada tally sheet
\item
  Tangkap satwa tersebut jika kondisi memungkinkan untuk dilakukan.

  \begin{itemize}
  \tightlist
  \item
    Katak: Perlahan dekati satwa hingga mencapai jangkauan tangan, posisikan tangan sejajar dengan badan katak dari belakang. Cekungkan tangan dan arahkan ke depan moncong katak dengan perlahan, kemudian dengan gerakan cepat menangkap katak dengan genggaman yang tidak terlalu kuat
  \item
    Ular: Jangan coba untuk menangkap ular jika tidak benar -- benar diketahui bahwa ular tesebut tidak berbisa. Gunakan graber atau tongkat untuk mengangkat ular kemudian letakkan di atas permukaan tanah. Perlahan tahan kepala ular dengan tongkat. Setelah tenang, genggam leher ular dengan posisi semua jari berada di bagian leher dan jempol menahan bagian atas kepala. Jangan mengambil risiko untuk menangkap ular berbisa seperti spesies--spesies dari famili Elapidae dan Viperidae. Cukup lakukan pengambilan foto dari beberapa bagian tubuh seperti punggung, sisi tubuh dan bagian atas, sisi kepala dari jarak yang relatif aman tanpa mengganggunya. Foto tersebut dapat dijadikan dokumentasi untuk keperluan identifikasi kemudian.\\
  \end{itemize}
\item
  Lakukan identifikasi jenis, ukur panjang badan (SVL) dan panjang ekor (Gambar \ref{fig:figmhf}. Untuk ular, gunakan graber atau tongkat untuk mengangkat ular kemudian letakkan di atas permukaan tanah. Perlahan tahan kepala ular dengan tongkat. Setelah tenang, genggam leher ular dengan posisi semua jari berada di bagian leher dan jempol menahan bagian atas kepala. Pengukuran panjang sebaiknya dilakukan dengan tetap meletakkan badan ular di atas tanah, sedangkan pengukuran menggunakan tali atau meteran jahit dengan mengikuti alur badan ular
\item
  Jika sudah dapat memastikan jenisnya dan selesai melakukan pengukuran, lepas kembali hewan tersebut ke tempat semula ditemukan.
\item
  Jika belum dapat teridentifikasi, masukkan hewan ke dalam plastik spesimen (untuk katak dan kodok), dan kantung kain (untuk reptil). Tulis kode jenis pada plastik dengan menggunakan spidol dengan tinta permanen (spidol waterproof), (catatan: satu plastik untuk satu individu), catat kode jenis di plastik pada tally sheet, dengan plastik spesimen. Catatan: jangan meniup plastik untuk memberikan udara.
\item
  Hentikan pengamatan saat:

  \begin{itemize}
  \tightlist
  \item
    Waktu pengamatan berakhir. Catat waktu akhir pengamatan, cuaca, dan kondisi lingkungan (suhu, kelembapan)
  \item
    Terjadi hujan lebat yang tidak memungkinkan pengamatan dilakukan, karena akan membahayakan pengamat saat melakukan pengamatan di sungai.
  \end{itemize}
\item
  Cek kembali kelengkapan alat, kondisi anggota tim dan sampel yang dibawa sebelum pulang.
\end{itemize}

\begin{figure}

{\centering \includegraphics[width=1\linewidth]{images/mhf_ilustration} 

}

\caption{Pengukuran Snout Vent Length (SVL) dan Tail Length (TL) pada spesimen herpetofauna; A) Katak B) Kadal, Cicak, Biawak C) Ular}\label{fig:figmhf}
\end{figure}

\hypertarget{foto-dan-koleksi-spesimen}{%
\subsubsection*{Foto dan koleksi spesimen}\label{foto-dan-koleksi-spesimen}}
\addcontentsline{toc}{subsubsection}{Foto dan koleksi spesimen}

Foto satwa berfungsi sebagai bukti keberadaan jenis tersebut, dokumentasi dan media identifikasi. Foto satwa terbagi menjadi dua berdasarkan kegunaannya, yaitu foto sebagai \emph{display} dengan menunjukkan momen dan posisi menarik dari satwa tersebut. Kemudian foto sebagai media identifikasi dengan menunjukkan detail dari bagian morfologi satwa (Gambar \ref{fig:spesamf1}). Pengambilan foto dapat dilakukan pada saat menemukan satwa di lokasi/ habitat alaminya, atau dengan menggunakan studio yang dibuat menyerupai habitat alami satwa tersebut. Perlu diperhatiakan dalam pengambilan foto harus memperhatiakan substrat latar belakang yang sesuai dengan habitatnya, kemampuan kamera dalam membuat foto makro dengan kondisi gelap (menggunakan \emph{blitz}), dan pembanding untuk foto identifikasi. Pemilihan kamera yang tepat terutama dengan kemampuan makro tinggi (jarak lensa dengan satwa kurang dari 10 cm), pengaturan ISO tinggi, mode pengambilan foto pada malam hari, dan pengaturan flash (manual) sangat menentukan hasil foto.

Jika jenis yang akan didokumentasikan (foto) merupakan jenis yang terlalu sulit untuk diambil di lokasi terutama untuk foto identifikasi, maka dapat dibuat studio buatan yang menyerupai habitatnya. Studio buatan dapat dibuat dengan serasah daun, batang pohon, batu, dan lain-lain pada suatu ruangan (kotak) yang tertutup. Foto identifikasi biasanya dilakukan pada spesimen yang telah diawetkan (mati), agar terlihat jelas morfologi ataupun anatomi tubuh jenis tersebut (gambar \ref{fig:spesamf1}).

Beberapa hal penting yang perlu diperhatikan dalam melakukan survei Herpetofauna, adalah sebagai berikut:

\begin{itemize}
\tightlist
\item
  Pengamatan dilakukan pada lokasi yang telah disurvei dengan data jalur, koordinat lokasi, penanda jalur yang lengkap dan perizinan kepada pengelola serta masyarakat sekitar lokasi.
\item
  Surveior minimal dua orang, sebaiknya didampingi oleh pemandu yang mengetahui kondisi lokasi dan mampu melakukan pengamatan malam.
\item
  Persiapkan perlengkapan lapang, alat komunikasi dan perlengkapan P3K untuk mengantisipasi terjebak di lokasi pengamatan dan terpaksa menginap.
\item
  Pertimbangkan kondisi cuaca dan kesehatan tim, jika tidak memungkinkan jangan dipaksakan untuk melakukan pengamatan.
\item
  Surveior sebaiknya melakukan pencarian dengan jarak yang dapat dipantau satu dengan yang lainnya.
\item
  Saat ditemukan satwa, sesegera mungkin mengambil data yang diperlukan dan melanjutkan pencarian.
\item
  Perlakukan satwa yang ditangkap dengan hati-hati tanpa menyakiti, lebih baik menggunakan tangan secara langsung untuk memegang satwa.
\item
  Penanganan pada jenis yang berbahaya dan sulit ditangkap sebaiknya dilakukan minimal dua orang.
\item
  Penyimpanan spesimen yang ditangkap, untuk amfibi ditempatkan pada plastik, sedangkan untuk reptil ditempatkan pada kantong kain dan satu individu satu kantong.
\item
  Pengamatan selesai jika jalur atau plot yang ditentukan sudah teramati, alokasi waktu sudah habis, perubahan kondisi cuaca dan medan yang membahayakan.
\item
  Secepat mungkin dilakukan pengambilan data pada spesimen yang ditangkap dan sesegera mungkin dilakukan preservasi/ pengawetan jika diperlukan.
\item
  Lakukan preservasi pada lokasi yang terpisah, bersih, gunakan masker dan sarung tangan karet (\emph{latex gloves}) dan jangan lupa mencuci tangan sebelum dan sesudah preservasi.
\item
  Penyimpanan spesimen yang diawetkan sebaiknya diletakkan pada tempat khusus yang terhindar dari cahaya matahari langsung.
\end{itemize}

\hypertarget{mamalia}{%
\section*{Mamalia}\label{mamalia}}
\addcontentsline{toc}{section}{Mamalia}

Pengamatan mamalia secara umum jauh lebih sulit dibandingkan dengan taksa lainnya dikarenakan di hutan hujan tropis, beberapa taksa dari mamalia cenderung sangat sulit untuk dijumpai, oleh karena itu metode yang digunakan lebih banyak mengandalkan penggunaan peralatan untuk penangkapan satwa dibandingkan dengan pengamatan atau perjumpaan secara langsung, bahkan pengamatan yang bersifat observatif seringkali hanya mengandalkan jejak yang ditinggalkan dari satwa itu sendiri.

Dikarenakan variasi perilaku dan relung yang jauh berbeda dari setiap taksa mamalia, maka teknik pengamatannya juga berbeda-beda. Beberapa jenis mamalia besar dapat dengan mudah dideteksi secara langsung, seperti jenis-jenis primata, tetapi sebagian besar mamalia kecil memerlukan pengukuran anggota tubuhnya sebelum dapat diidentifikasi. Beberapa buku menjabarkan metode yang spesifik untuk satwa bahkan spesies tertentu seperti pengamatan orangutan (Atmoko \& Rifqi, 2012), kelelawar (Prasetyo et al., 2011) maupun harimau dan satwa mangsanya (Pinondang et al., 2018).

Dalam konteks kajian yang biasa dilakukan oleh FFI`s IP biasanya menggunakan metode pengamatan langsung menggunakan transek garis yang dikombinasikan dengan metode eksplorasi, sedangkan untuk metode penangkapan, dapat menggunakan \emph{live trap}, jaring kabut, perangkap harpa hingga kamera pengintai (\emph{camera trap}). Metode menggunakan kamera pengintai akan dijabarkan dalam panduan khusus, namun hasil yang didapat dari kamera pengintai dapat dikombinasikan dengan kajian ini untuk melengkapi daftar spesies, terutama jenis-jenis yang elusif

Target taksa dari setiap metode terangkum dalam \sout{Tabel 1}. Dalam praktiknya, kajian keanekaragaman hayati mamalia, tidak perlu menggunakan semua perangkat yang disebutkan. Sesuaikan perangkat yang akan dibawa dengan tujuan survei serta sumber daya yang tersedia.

\hypertarget{persiapan-tim-2}{%
\subsection*{Persiapan Tim}\label{persiapan-tim-2}}
\addcontentsline{toc}{subsection}{Persiapan Tim}

Pelaksanaan survei mamalia di FFI`s IP membutuhkan keterampilan dan langkah kerja yang beragam, sehingga membutuhkan beberapa asisten untuk membantu di lapangan. Idealnya, satu orang peneliti utama dibantu dengan 4 orang asisten dengan peran dan tanggung jawab seperti terlihat pada Tabel \ref{tab:tbtm}. Dalam pelaksanaannya, jumlah, susunan, formasi, peran dan tanggung jawab asisten dapat berubah sesuai kondisi.

\begin{longtable}[]{@{}
  >{\raggedright\arraybackslash}p{(\columnwidth - 4\tabcolsep) * \real{0.1765}}
  >{\raggedright\arraybackslash}p{(\columnwidth - 4\tabcolsep) * \real{0.4118}}
  >{\raggedright\arraybackslash}p{(\columnwidth - 4\tabcolsep) * \real{0.4118}}@{}}
\caption{\label{tab:tbtm} Peran dan tanggung jawab tim mamalia}\tabularnewline
\toprule()
\begin{minipage}[b]{\linewidth}\raggedright
Peran
\end{minipage} & \begin{minipage}[b]{\linewidth}\raggedright
Tanggung jawab
\end{minipage} & \begin{minipage}[b]{\linewidth}\raggedright
Syarat khusus
\end{minipage} \\
\midrule()
\endfirsthead
\toprule()
\begin{minipage}[b]{\linewidth}\raggedright
Peran
\end{minipage} & \begin{minipage}[b]{\linewidth}\raggedright
Tanggung jawab
\end{minipage} & \begin{minipage}[b]{\linewidth}\raggedright
Syarat khusus
\end{minipage} \\
\midrule()
\endhead
Pengamat utama & Koordinasi pekerjaan kepada seluruh asisten, memastikan seluruh data tercatat dengan benar sesuai protokol pada lembar data & Memahami protokol survei mamalia dan identifikasi jenis mamalia \\
Asisten I - III & Memasang dan membongkar perangkap harpa dan jaring kabut & Dapat memasang dan membongkar perangkap harpa dan jala kabut, serta mampu memisahkan kelelawar yang terperangkap ke dalam kantong sampel \\
Asisten V & Memasang dan membongkar perangkap kasmin & Dapat memasang dan membongkar perangkap kasmin, serta mampu memisahkan hewan yang terperangkap ke dalam kantong sampel \\
\bottomrule()
\end{longtable}

\hypertarget{peralatan-2}{%
\subsection*{Peralatan}\label{peralatan-2}}
\addcontentsline{toc}{subsection}{Peralatan}

\hypertarget{protokol-pengamatan-2}{%
\subsection*{Protokol Pengamatan}\label{protokol-pengamatan-2}}
\addcontentsline{toc}{subsection}{Protokol Pengamatan}

\hypertarget{transek-garis}{%
\subsubsection*{Transek garis}\label{transek-garis}}
\addcontentsline{toc}{subsubsection}{Transek garis}

Pada metode transek garis, pengamat berjaringn di sepanjang jalur transek sambil mengidentifkasi dan mencatat satwa yang dijumpai secara langsung atau tanda satwa yang ditinggalkan seperti misalnya: tapak, cakaran, feses, aroma, atau sarang, yang dapat digunakan sebagai indikator keberadaan suatu spesies. Metode ini biasanya digunakan untuk mendeteksi mamalia arboreal secara umum dan beberapa jenis mamalia terrestrial berukuran sedang hingga besar.

Jejak satwa yang ditinggalkan oleh mamalia biasanya lebih mudah dideteksi pada lokasi spesifik. Tapak, seringkali mudah ditemukan di tanah yang berlumpur, seperti kubangan atau dekat sungai tempatnya berhenti untuk minum. Cakaran, dapat ditemukan pada batang pohon hidup maupun yang sudah mati. Aroma khas, biasanya tercium berasal dari air seni yang jatuh ke pepohonan, misalnya aroma pandan sebagai indikasi adanya musang. Sarang, yang dibuat oleh mamalia biasanya berukuran besar dan terletak pada cabang-cabang atau kanopi pohon, misalnya sarang orangutan. Beberapa panduan dapat dijadikan bacaan awal untuk referensi identifikasi jejak satwa liar, seperti Strien (1983) untuk identifikasi tapak mamalia, Chame (2003) untuk morfologi feses dan Atmoko \& Rifqi (2012) untuk bentuk dan contoh sarang orangutan.

Teknik mengidentifikasi satwa dari jejak yang ditinggalkan perlu dilatih secara berkala bersama surveior yang sudah berpengalaman. Menggunakan orang lokal yang biasa berburu atau secara rutin menjelajah hutan juga dapat sangat membantu dalam mengenali jejak dan suara satwa. kemampuan membuat catatan lapangan saat menemukan jejak sangat diperlukan untuk bisa mendokumentasikan dengan baik setiap temuan dan menjadi pembelajaran bersama bagi para surveior di kemudian hari.

\textbf{Cara Pelaksanaan:}

\begin{enumerate}
\def\labelenumi{\arabic{enumi}.}
\tightlist
\item
  Pengamatan di transek garis dilakukan pada pagi hari pukul 06:00 dan berjaringn secara perlahan hingga ke ujung transek sejauh 2000 m, kemudian melakukan pengamatan lagi pada malam hari. Pengamatan malam dapat dimulai pada pukul 16:00 jika bertujuan untuk sambil memasang perangkap, atau jam 19:00 jika hanya melakukan transek garis.
\item
  Berjaringn secara perlahan dan konsisten, serta usahakan sesunyi mungkin sambil memperhatikan keberadaan satwa liar.
\item
  Pengamat utama harus membagi tim untuk personil yang khusus mencari satwa secara langsung di tajuk-tajuk pohon untuk mendeteksi primata dan satwa arboreal lainnya. Personil ini harus selalu berjaringn sepanjang jalur utama transek dan berada di barisan paling depan.
\item
  Personil lainnya ditugaskan untuk mencari jejak satwa di lantai hutan dan batang pohon. Personil ini dianjurkan untuk menjelajahi area disekitar jalur transek bila memungkinkan.
\item
  Apabila melihat satwa secara langsung, usahakan catat jarak perpendikular satwa tersebut dari jalur transek (Gambar \ref{fig:figppd}), kemudian catat temuan tersebut di lembar data mamalia (Gambar \ref{fig:ldmtg}). Apabila kondisi memungkinkan, berhenti sejenak untuk dapat mengambil foto satwa tersebut.
\item
  Apabila menemukan jejak satwa berupa tapak, kotoran, cakaran dan jejak visual lainnya catat di lembar data dan foto temuan tersebut menggunakan ukuran pembanding (Gambar \ref{fig:pugm}).
\item
  Apabila tim mendengar suara mamalia yang belum dikenali atau diragukan spesiesnya, usahakan berhenti sejenak untuk merekam suara tersebut supaya dapat diidentifikasi lebih lanjut.
\item
  Pastikan seluruh pencatat temuan sesuai dengan lembar data mamalia
\end{enumerate}

\begin{figure}

{\centering \includegraphics[width=1\linewidth]{images/ppd_ilustration} 

}

\caption{Ilustrasi jarak perpendikular terhadap transek}\label{fig:figppd}
\end{figure}

\begin{figure}

{\centering \includegraphics[width=1\linewidth]{images/pugmark} 

}

\caption{Ilustrasi pengambilan gambar jejak dengan ukuran pembanding (sumber: doi:10.1371/journal.pone.0172065.g003 )}\label{fig:pugm}
\end{figure}

Pada penerapannya, pengamatan langsung seringkali dikombinasikan dengan metode eksplorasi, yang tidak dibatasi oleh jalur transek. Seperti perjaringnan menuju kamp atau perjaringnan antar transek. Pengamat ini dilakukan dengan mencari mamalia secara intuitif, seperti memeriksa kedalam lubang kayu mati untuk melihat kelelawar, memeriksa sempadan sungai untuk melihat jejak kotoran berang-berang dan lainnya. Setiap temuan juga dapat dimasukan ke dalam lembar data sebagai pelengkap informasi keanekaragaman spesies dalam suatu areal kajian.

\hypertarget{live-trap}{%
\subsubsection*{\texorpdfstring{\emph{Live trap}}{Live trap}}\label{live-trap}}
\addcontentsline{toc}{subsubsection}{\emph{Live trap}}

Perangkap ini digunakan untuk menangkap mamalia kecil (cth; tikus, bajing, tupai), disebut live trap karena satwa yang terangkap biasanya dilepaskan kembali. Di Indonesia, umumnya ada dua versi yaitu, perangkap kasmin yang paling umum didapatkan (Gambar \ref{fig:lvtrap} (a)), dan perangkap sherman (Gambar \ref{fig:lvtrap} (b)). Kedua perangkap tersebut harus dimodifikasi supaya dapat dilipat dan mudah dibawa dengan ukuran sekitar 13 x 26 x 13 cm. Pada beberapa penelitan, perangkap ini juga bisa dimodifikasi untuk dipasang di batang pohon (Gambar \ref{fig:stot}).

\begin{figure}

{\centering \includegraphics[width=1\linewidth]{images/livetrap} 

}

\caption{(a) perangkap kasmin (b) perangkap sherman (sumber; shermantraps.com)}\label{fig:lvtrap}
\end{figure}

\begin{figure}

{\centering \includegraphics[width=1\linewidth]{images/stontrees} 

}

\caption{Perangkap sherman yang dipasang diatas pohon (sumber; Díaz-N. et al., 2011)}\label{fig:stot}
\end{figure}

\textbf{Cara Pelaksanaan :}

\begin{enumerate}
\def\labelenumi{\arabic{enumi}.}
\tightlist
\item
  Sebelum dibawa ke area survei, perangkap harus disterilkan dulu dengan dicuci atau di rendam beberapa saat dengan air panas untuk menghilangkan bau dari pemakaian sebelumnya.
\item
  Pastikan seluruh mekanisme perangkap dalam keadaan baik dengan memicu pelatuk tempat umpan serta penutup perangkap dapat menutup dengan utuh.
\item
  Pada saat di area survei, letakan perangkap di tempat-tempat yang diduga menjadi tempat tinggal atau jalur yang dilewat mamalia kecil, seperti di dekat lubang yang berada di tanah, pohon mati berlubang, di bawah pohon, bekas makan dan lainnya. Perangkap dapat dibawa dan dipasang saat melakukan pengamatan pada transek di pagi atau di malam hari.
\item
  Kamuflasikan perangkap dengan seresah dan pasang sekokoh mungkin sejajar dengan permukaan tanah supaya hewan dapat berjaringn secara alami kedalam perangkap lalu tandai lokasi pengamatan dengan pita atau label berwarna terang di ranting atau kayu sejajar dengan mata supaya mudah ditemukan saat pengecekan atau pengambilan.
\item
  Masukan umpan ke dalam perangkap. Umpan yang digunakan bervariasi, mulai dari singkong atau kentang yang diolesi selai kacang, ikan asin yang sedikit dibakar, potongan ikan atau daging, bahkan ada juga yang menaruh jangkring atau belalang didalam kantung kain. Pada dasarnya dibutuhkan suatu umpan yang dapat mengeluarkan aroma, atau suara yang dapat memancing tikus mendatangi perangkap.
\item
  Catat lokasi perangkap dan umpan yang digunakan ke dalam lembar data perangkap (Gambar \ref{fig:ldmpk})
\item
  Perangkap dipasang minimal 1x24 jam untuk kemudian dievaluasi untuk tetap berada dilokasi yang sama atau dipindahkan. Namun setiap perangkap wajib diperiksa pada pagi dan malam hari. Pada saat pengecekan, usahakan ganti umpan jika kondisinya sudah tidak bagus.
\item
  Bila ada satwa yang tertangkap, keluarkan menggunakan sarung tangan yang tebal, lalu masukan ke dalam kantong kain untuk melakukan pengukuran bagian tubuh di kamp. Bagian tubuh yang biasa diukur adalah panjang tubuh (HB), panjang kaki belakang (HF) dan panjang ekor (T) (Gambar \ref{fig:mfsm}). Jika sudah selesai diidentifikasi dan difoto, maka lepaskan kembali hewan di tempat tertangkap.
\end{enumerate}

\begin{figure}

{\centering \includegraphics[width=1\linewidth]{images/mfm_ilustration} 

}

\caption{Ilustrasi pengukuran bagian tubuh untuk mamalia kecil}\label{fig:mfsm}
\end{figure}

\hypertarget{jaring-kabut-dan-perangkap-harpa}{%
\subsubsection*{Jaring kabut dan perangkap harpa}\label{jaring-kabut-dan-perangkap-harpa}}
\addcontentsline{toc}{subsubsection}{Jaring kabut dan perangkap harpa}

Perangkap jaring kabut dan harpa dapat digunakan untuk menangkap kelelawar, kedua perangkap tersebut dapa digunakan bersama-sama atau sendiri-sendiri. Pada dasarnya perangkap jaring kabur lebih efektif untuk menangkap kelelawar dalam kelompok Megachiroptera atau kelelawar pemakan buah, sedangkan perangkap harpa untuk menangkap kelompok Microchiroptera atau kelelawar pemakan serangga.

\textbf{Jaring kabut:}

\begin{itemize}
\tightlist
\item
  Jaring kabut dipasang sepanjang jalur transek yang diduga menjadi perlintasan keleawar seperti tepi hutan, pintu hutan, melintangi sungai, punggungan jalur dan daerah terbuka. Pemasangan dapa juga mempertimbangkan area dengan pepohonan yang sedang berbuah dan berbunga (Gambar \ref{fig:ldmp})
\item
  Jaring kabut idealnya dipasang pada sore hari sebelum matahari terbenam.
\item
  Jaring kabut biasanya dipasang menggunakan ranting atau batang kayu pada jarak 0.5 -- 3-meter dari permukaan tanah. Pemasangan jaring kabut harus longgar pada setiap segmen, supaya kelelawar tidak terpantul keluar (Gambar \ref{fig:mistnet}C)
\item
  Setelah jaring kabut terpasang, catat lokasi pemasangan kedalam lembar data (Gambar \ref{fig:ldmp}). Periksa kembali perangkap pada malam hari saat melakukan pengamatan malam hingga maksimal sampai jam 24:00. Apabila terdapat lebih dari 10 individu, perangkap dapat ditutup, namun jika kurang dari itu perangkap dapat dibiarkan terpasang sampai pagi.
\item
  Perangkap yang dibiarkan terpasang sampai pagi di evaluasi untuk tetap dipertahankan atau dipindahkan ke lokasi yang lain. Apabila perangkap ingin dilanjutkan untuk malam berikutnya, tutup perangkap dengan menggulung bagian tengahnya supaya tidak ada burung yang terjebak.
\item
  Untuk melepaskan kelelawar yang terperangkap, periksa dulu arah masuknya kelelawar pada jaring kabut
\item
  Kendalikan terlebih dahulu kepalanya (bila terlalu sulit, bisa menggunakan kayu untuk gumpalan kain untuk mengalihkan perhatian), setelah itu pegang bagian badannya, lalu keluarkan mulai dari bagian kaki, setelah itu sayap dan kepala.
\end{itemize}

\begin{figure}

{\centering \includegraphics[width=1\linewidth]{images/mistnetuses} 

}

\caption{Ilustrasi pemasangan jaring kabut; (a) Pintu hutan (b) diatas sungai dan, (c) detil pemasangan jaring kabut yang longgar di setiap segmen (sumber; (Borisenko and Kruskop, 2003))}\label{fig:mistnet}
\end{figure}

\textbf{Perangkap harpa:}

\begin{itemize}
\tightlist
\item
  Perangkap harpa dipasang pada jalur transek, dekat sungai, mulut gua, daerah ekoton atau pinggiran hutan.
\item
  Pemasangan dilakukan sebelum matahari terbenam, sekitar jam 16:00 -- 17:00, sesuaikan lama pemasangan dengan waktu pemasangan.
\item
  Tutup bagian kiri, kanan dan atasnya perangkap harpa dengan menggunakan tumbuhan disekitar untuk menghindari kelelawar melalui bagian terbuka
\item
  Tutup bagian kiri, kanan dan atasnya kantung harpa menggunakan daun, karena beberapa jenis kelelawar mampu memanjat keluar melalui bagian tersebut. Setelah terpasang dengan sempurna, catat lokasi pemasangan kedalam lembar data (Gambar \ref{fig:ldmp}).
\item
  Perangkap harpa dapat ditinggal hingga pagi hari untuk memeriksa kelelawar yang tertangkap kemudian dibongkar dan dipindahkan ke lokasi lainnya.
\end{itemize}

\textbf{Pencatatan kelelawar:}

Kelelawar yang tertangkap baik dari dari jaring kabut dan perangkap harpa dapat disimpan dulu ke dalam kantong kain untuk dilakukan identifikasi, pengambilan foto dan pengukuran tubuh (Gambar \ref{fig:kelmor}) di kamp. Bagian tubuh yang diukur adalah panjang tubuh (HB), panjang lengan depan (FA), panjang kaki belakang (HF), panjang ekor (T), panjang telinga (E) dan tragus (Tr) bila ada. Setelah selesai, lepaskan kembali kelelawar di lokasi tertangkap pada malam atau pagi hari.

\begin{figure}

{\centering \includegraphics[width=1\linewidth]{images/kfm_ilustration} 

}

\caption{Ilustrasi pengukuran bagian tubuh kelelawar (sumber; Borisenko and Kruskop, 2003)}\label{fig:kelmor}
\end{figure}

\hypertarget{vegetasi}{%
\section*{Vegetasi}\label{vegetasi}}
\addcontentsline{toc}{section}{Vegetasi}

Secara umum, survei vegetasi dan pengukuran biomassa yang dilakukan FFI`s IP bertujuan untuk menggambarkan kondisi hutan melalui analisis vegetasi (jenis tumbuhan dominan, kerapatan pohon, tutupan kanopi pohon, dsb) serta potensi cadangan karbon yang terkandung di hutan tersebut (karbon atas permukaan, bawah permukaan, serasah, dan pohon mati). Analisis vegetasi biasanya digunakan untuk menggambarkan struktur dan komposisi dari vegetasi suatu habitat. Selain itu, sekaligus juga dapat digunakan untuk inventarisasi biodiversitas floristik suatu area.

Dari hasil survei vegetasi, kandungan karbon (karbon permukaan dan bawah) dapat di estimasi dengan menggunakan persamaan alometrik yang paling sesuai (Krisnawati et al., 2012; SNI 7724:2011; SNI 7725:2011). Persamaan alometrik yang digunakan akan menggambarkan biomassa dari tiap jenis pohon yang di data. Biomassa tersebut kemudian dikalikan dengan faktor pengali standar, 47\% (IPCC, 2012), untuk memperoleh kandungan karbon dari tiap jenis pohon yang di data. Kandungan karbon tersebut kemudian dikalikan dengan berat molekul CO2 (3.67) untuk menghasilkan potensi emisi CO2 dari tiap jenis pohon yang di data. Pertumbuhan pohon atau riap dapat dihitung dengan menggunakan data lokal dari plot permanen, atau menggunakan asumsi pertumbuhan sebesar 3,4-ton biomassa per hektar per tahun (Eggleston et al., 2006; Penman et al., 2003).

\textbf{Batasan Studi}

survei vegetasi yang dilakukan FFI`s IP dibatasi pada tingkat tumbuhan tinggi tegakan kayu dengan ukuran kayu yang diukur dimulai dari diameter setinggi dada (DBH) 5 cm, dan diklasifikasikan ke dalam 3 kelas, yaitu:

\begin{itemize}
\tightlist
\item
  Kelas C: Tiang dan pancang, DBH 5 -- 14,99 cm
\item
  Kelas B: Pohon sedang, DBH 15 -- 29,99 cm
\item
  Kelas A: Pohon besar, DBH ≥ 30 cm
\end{itemize}

Liana dan pohon perambat tidak masuk dalam kategori di atas karena tidak berbentuk tegakan. Pembatasan ini dilakukan untuk menyesuaikan efektifitas usaha dan waktu dimana sedapat mungkin objek yang disurvei merupakan komunitas atau struktur utama pembentuk vegetasi. Selain itu, data pengukuran botani ini digunakan pula untuk penghitungan karbon tegakan (Above Ground Biomass - AGB) dimana kandungan karbon terbesar berada pada ketiga kelas kayu tersebut. Karbon pada tingkat semaian dan herba tidak berpengaruh secara signifikan terhadap nilai cadangan karbon

\hypertarget{persiapan-tim-3}{%
\subsection*{Persiapan Tim}\label{persiapan-tim-3}}
\addcontentsline{toc}{subsection}{Persiapan Tim}

Pelaksanaan survei vegetasi di FFI`s IP membutuhkan ketelitian dan langkah kerja yang cukup banyak sehingga membutuhkan beberapa asisten untuk membantu di lapangan. Idealnya, satu orang peneliti utama dibantu dengan 4 orang asisten lokal dengan peran dan tanggung jawab seperti terlihat pada Tabel \ref{tab:tabtmv}. Dalam pelaksanaannya, jumlah, susunan dan formasi peran dan tanggung jawab asisten lokal dapat berubah sesuai kondisi.

\begin{longtable}[]{@{}
  >{\raggedright\arraybackslash}p{(\columnwidth - 4\tabcolsep) * \real{0.2941}}
  >{\raggedright\arraybackslash}p{(\columnwidth - 4\tabcolsep) * \real{0.4118}}
  >{\raggedright\arraybackslash}p{(\columnwidth - 4\tabcolsep) * \real{0.2941}}@{}}
\caption{\label{tab:tabtmv} Peran dan tanggung jawab tim vegetasi}\tabularnewline
\toprule()
\begin{minipage}[b]{\linewidth}\raggedright
Peran
\end{minipage} & \begin{minipage}[b]{\linewidth}\raggedright
Tanggung jawab
\end{minipage} & \begin{minipage}[b]{\linewidth}\raggedright
Syarat khusus
\end{minipage} \\
\midrule()
\endfirsthead
\toprule()
\begin{minipage}[b]{\linewidth}\raggedright
Peran
\end{minipage} & \begin{minipage}[b]{\linewidth}\raggedright
Tanggung jawab
\end{minipage} & \begin{minipage}[b]{\linewidth}\raggedright
Syarat khusus
\end{minipage} \\
\midrule()
\endhead
Peneliti utama & Koordinasi pekerjaan kepada seluruh asisten, memastikan seluruh data tercatat dengan benar sesuai protokol pada lembar data & Paham dengan protokol survei botani dan pengukuran biomassa \\
Asisten I: Pengenal jenis & Memberikan informasi tentang nama lokal dan informasi yang dibutuhukan mengenai jenis pohon & Mengenal pohon dan nama lokalnya dengan baik \\
Asisten II: Pembuat petak & Pengarah kompas dan membuat garis petak menggunakan meteran & Bisa menggunakan kompas dengan baik dan teliti \\
Asisten III: Pembuat petak & Pembuat garis petak & Paham menggunakan meteran standar \\
Asisten IV: Pengukur & Mengukur diameter pohon dan tinggi pohon & Terampil menggunakan diameter tape, distometer, serta ahli memanjat pohon \\
Asisten V: pengukur dan pengambil spesimen & Memasang plat alumunium pada pohon dan mengambil sampel & Ahli memanjat pohon untuk mengambil sampel \\
\bottomrule()
\end{longtable}

\hypertarget{peralatan-3}{%
\subsection*{Peralatan}\label{peralatan-3}}
\addcontentsline{toc}{subsection}{Peralatan}

Peralatan dan bahan yang digunakan dalam survey vegetasi terangkum pada Tabel \ref{tab:tabpv}

\begin{longtable}[]{@{}
  >{\raggedright\arraybackslash}p{(\columnwidth - 4\tabcolsep) * \real{0.2941}}
  >{\raggedright\arraybackslash}p{(\columnwidth - 4\tabcolsep) * \real{0.4118}}
  >{\raggedright\arraybackslash}p{(\columnwidth - 4\tabcolsep) * \real{0.2941}}@{}}
\caption{\label{tab:tabpv} Peralatan yang dibutuhkan tim vegetasi}\tabularnewline
\toprule()
\begin{minipage}[b]{\linewidth}\raggedright
Peralatan
\end{minipage} & \begin{minipage}[b]{\linewidth}\raggedright
Penggunaan
\end{minipage} & \begin{minipage}[b]{\linewidth}\raggedright
Spesifikasi
\end{minipage} \\
\midrule()
\endfirsthead
\toprule()
\begin{minipage}[b]{\linewidth}\raggedright
Peralatan
\end{minipage} & \begin{minipage}[b]{\linewidth}\raggedright
Penggunaan
\end{minipage} & \begin{minipage}[b]{\linewidth}\raggedright
Spesifikasi
\end{minipage} \\
\midrule()
\endhead
Diameter tape 3M (x 2) dan 1 M (x1) & Pengukuran diameter pohon & Bahan \emph{fiber} dan angka tidak mudah pudar \\
Meteran pengukur 100 M (x1) dan 30 M (x1) & Pengukuran batas-batas petak & Bahan \emph{fiber glass}, angka tidak mudah pudar \\
Tali rapia/ flagging tape & Penanda garis-garis atau titik batas. Khusus rapia, setelah penggunaan harus dilepas kembali. & Berwarna mencolok, awet dan \emph{biodegradable} \\
Densiometer & Pendugaan persentase tutupan tajuk (\%) & - \\
Kompas & Pembuatan petak & Stabil dan dapat berfungsi pada bidang miring \\
Pasak besi & sebagai penanda titik yang ditanam di dalam tanah, untuk memudahkan pendeteksian titik menggunakan metal detector & Berbahan besi \\
GPS (x2) & Penandaan lokasi geografis & Tahan banting, tahan air, dengan akurasi sekitar 3 m \\
Binoculars & Melihat rinci daun tajuk & Pembesaran minimal 20 x \\
Laser distance meter & Pengukuran tinggi pohon & - \\
Kamera digital (\emph{prosummer}/SLR) & Foto spesimen, rona lingkungan, karakter tumbuhan & Perbesaran lensa min. 25x, memiliki fungsi \emph{macro} dan \emph{stabilizer} \\
Parang & Menyiangi tumbuhan bawah untuk jalur, membuat takikan batang & Ramping, tajam, panjang kurang lebih 14-16 inchi \\
Plastik sampel & Mengumpulkan koleksi spesimen tumbuhan & Ukuran besar, kira-kira 40 x 60 cm \\
Pelat alumunium (+ paku \& palu) & Penanda fisik pohon & Kuat, ringan, tidak mudah rapuh \\
Lembar data & Lembar pencatatan data & - \\
Alat Tulis (Pensil, Spidol marker, papan dada) & Pencatatan data dan penandaan & Kuat, tidak mudah luntur \\
\bottomrule()
\end{longtable}

\hypertarget{protokol-pengamatan-3}{%
\subsection*{Protokol Pengamatan}\label{protokol-pengamatan-3}}
\addcontentsline{toc}{subsection}{Protokol Pengamatan}

Survei vegetasi dilakukan dengan metode petak persegi berukuran 125 x 20 m merujuk pada SNI 7724:2011 dan P.33/Menhut-II/2009. Petak diletakkan pada garis transek dengan interval antar petak 500 m penelitian yang telah ditetapkan. Setiap petak terdiri dari tiga subpetak dengan pembagian kategori kelas pohon yang berbeda (Tabel \ref{tab:tabpvil}). Penempatan petak persegi terhadap transek diupayakan berseling kanan dan kiri per jarak 500 m antar titik awal petak. Petak pertama ditempatkan di sebelah kanan jalur transek. Penempatan sub-petak B dan C adalah di dalam dan sebelah awal dari petak A, kemudian sub-petak C berada di dalam sub-petak B pada sisi yang menjauh garis transek. Ilustrasi penempatan petak dan pembagian kelas sub-petak dapat dilihat pada gambar Gambar \ref{fig:figpv}.

\begin{longtable}[]{@{}llll@{}}
\caption{\label{tab:tabpvil} Petak dan anak-petak persegi untuk tiap kategori vegetasi.}\tabularnewline
\toprule()
Ukuran sub-petak & DBH & Kategori & Kelas \\
\midrule()
\endfirsthead
\toprule()
Ukuran sub-petak & DBH & Kategori & Kelas \\
\midrule()
\endhead
10 m x 10 m & 5 - 15 cm & Tiang \& pancang & C \\
20 m x 20 m & 15 - 30 cm & Pohon kecil & B \\
20 m x 125 m & \textgreater{} 30 cm & Pohon Besar & A \\
\bottomrule()
\end{longtable}

\begin{figure}

{\centering \includegraphics[width=1\linewidth]{images/pv_ilustration} 

}

\caption{Bentuk dan peletakan petak persegi terhadap jalur transek}\label{fig:figpv}
\end{figure}

Alur kerja survei di lapangan untuk vegetasi dan biomassa pohon adalah sebagai berikut:

\begin{figure}

{\centering \includegraphics[width=1\linewidth]{images/av_ilustration} 

}

\caption{Diagram alur kerja survei vegetasi}\label{fig:figav}
\end{figure}

Rincian prosedur untuk setiap langkah alur di atas dijelaskan di bawah ini.

\hypertarget{penentuan-titik-petak}{%
\subsubsection*{Penentuan titik petak}\label{penentuan-titik-petak}}
\addcontentsline{toc}{subsubsection}{Penentuan titik petak}

Alur pengerjaan pertama dari survey botani adalah penentuan letak petak persegi. Idealnya, dan harus diupayakan, petak persegi ditempatkan tepat di titik-titik yang telah ditentukan (Gambar \ref{fig:figpv}). Namun kondisi aktual di lapangan kadang mendapati area yang tidak memungkinkan untuk membuat petak di titik tersebut, diantaranya medan yang sangat terjal sehingga membahayakan nyawa saat pengerjaan.

Langkah pengerjaan penentuan titik ini adalah sebagai berikut:

\begin{itemize}
\tightlist
\item
  Ketika tiba di titik yang ditetapkan, amati kondisi dimana petak akan dibuat.
\item
  Tentukan skenario yang akan digunakan, yaitu:
\item
  skenario A: bila kondisi memungkinkan, buat petak tepat pada titik yang telah ditetapkan.
\item
  skenario B: bila kondisi kurang memungkinkan terutama karena kondisi kontur terlalu terjal, maka pindahkan petak ke area yang cukup aman namun tetap pada garis transek.
\item
  Ketika letak petak sudah ditetapkan, rekam koordinat geografis petak menggunakan GPS tepat di titik awal atau titik petak 0,0, kemudian beri nama sesuai kode petak (plot ID).
\end{itemize}

\textbf{Hal yang perlu diperhatikan:}

Penentuan titik petak ini merupakan tanggung jawab dari surveior utama/ketua tim. Bila terjadi pemindahan titik petak harap catat pemindahannya, derajat arah panjang dan lebar petak, serta alasan pemindahan tersebut. Kondisi-kondisi yang memungkinkan untuk pemindahan letak petak diantaranya: kontur terlalu terjal/curam, dan terdapat sungai/danau tepat di titik awal atau di tengah petak yang dapat membiaskan luas petak secara signifikan. Untuk survey keanekaragaman, bila petak yang ditentukan berada pada area terbuka/terdegradasi, maka penempatan petak tetap sesuai dengan titik yang telah ditetapkan.

\hypertarget{pembuatan-petak-persegi}{%
\subsubsection*{Pembuatan petak persegi}\label{pembuatan-petak-persegi}}
\addcontentsline{toc}{subsubsection}{Pembuatan petak persegi}

Ukuran dan dimensi petak merupakan bagian penting dari rangkaian survey botani vegetasi dan biomassa pohon. Dimensi dan ukuran yang tidak sempurna akan menghasilkan bias terhadap data. Perlu diperhatikan baik-baik bahwa ukuran harus tepat menyesuaikan kontur permukaan tanah. Dalam hal ini, pembuatan petak dapat dilakukan dengan dua cara berdasarkan kondisi konturnya.

\begin{itemize}
\tightlist
\item
  \textbf{Kontur tanah cenderung rata.} Bila kontur permukaan tanah dimana petak akan dibuat cenderung rata, maka pembuatan petak dapat dilakukan dengan pembuatan batas-batas petak secara langsung sesuai Gambar \ref{fig:figpv} di atas.
\item
  \textbf{Kontur permukaan petak bergelombang / tidak rata.} Kontur permukaan yang tidak rata, umumnya di area berbukit, akan membuat bias dimensi dan ukuran sehingga luasan petak akan berubah. Petak tidak akan berbentuk persegi secara sempurna karena terpengaruh oleh perbedaan kontur antar batas petak. Untuk menyiasati hal ini, maka pembuatan petak dilakukan dengan teknik tulang ikan. Petak ini dibuat dengan terlebih dahulu membuat garis tengah petak, lalu mengukur batas-batas kiri dan kanan petak masing-masing sejauh 10 meter dari garis tengah. Dengan demikian, meski bentuk petak tidak persegi sempurna, namun ukuran dan luas petak tidak akan berubah. Bentuk dan dimensi petak yang akan dibuat idealnya seperti pada Gambar \ref{fig:figtiv}
\end{itemize}

\begin{figure}

{\centering \includegraphics[width=1\linewidth]{images/tiv_ilustration} 

}

\caption{Bentuk dan ukuran petak dan sub-petak dengan teknik tulang ikan}\label{fig:figtiv}
\end{figure}

Idealnya arah panjang petak menyesuaikan arah transek yaitu dari utara ke selatan atau sebaliknya. Pembuatan petak tulang ikan yang dijelaskan di bawah ini mengambil asumsi arah petak mengarah ke utara dan lebar ke arah timur. Langkah demi langkah pembuatan petak tulang ikan dijelaskan sebagai berikut:

\begin{enumerate}
\def\labelenumi{\alph{enumi})}
\item
  Buat tanda patok di titik (0,0) dan beri tanda fisik menggunakan flagging tape secukupnya, kemudian tanam besi dalam tanah di titik ini,
\item
  Dari titik (0,0) tembak arah 90°-garis panjang (ke arah lebar) menggunakan kompas, kemudian tarik meteran ukur sejauh 10 meter ke arah tersebut, beri tanda dengan flagging tape dan beri nama 10,0,
\item
  dari titik (10,0) tembak arah 0° / ke arah panjang p etak, lalu tarik garis 10 meter ke arah tersebut dan beri patok, titik ini adalah (10,10),
\item
  dari titik (10,10) buat garis ke samping kanan dan kiri tegak lurus garis, masing-masing 10 meter dan beri tanda flagging tape,
\item
  dari titik (10,10) buat garis ke arah panjang sejauh 10 meter (10,20) lalu lakukan langkah yang sama seperti poin `c' dan `d',
\item
  lakukan pengulangan seterusnya ke arah panjang petak hingga meter ke 120 atau titik (10,120),
\item
  dari titik (10,120) buat garis ke arah panjang sejauh 5 meter (10,125), tandai, kemudian lakukan langkah sama seperti tahap `d',
\item
  Petak telah selesai dibuat dan pastikan bahwa besi telah tertanam di titik 0,0; 10,0 ; 20,0; 10,10; dan 20,10.
\end{enumerate}

\textbf{Keterangan penting:}

Arah panjang transek di lapangan tidak selalu mengarah ke utara, pada kondisi tertentu arah transek bisa menyamping atau menyerong dari arah utara. Kunci ketepatan pembuatan batas-batas petak adalah ketelitian dalam membaca kompas, upayakan agar pembaca kompas teliti betul. Setiap survey yang telah dilakukan, besar kemungkinan akan dilakukan pemantauan, pengukuran ulang, atau verifikasi audit petak, pendeteksian terhadap titik awal petak menjadi sangat penting untuk memudahkan ditemukannya kembali petak-petak yang telah dibuat. Untuk menyiasati ini, tanda fisik dengan flagging tape dan besi yang ditanam jangan sampai hilang.

\hypertarget{identifikasi-rona-lingkungan}{%
\subsubsection*{Identifikasi rona lingkungan}\label{identifikasi-rona-lingkungan}}
\addcontentsline{toc}{subsubsection}{Identifikasi rona lingkungan}

Identifikasi lingkungan dilakukan untuk melihat gambaran rona hutan di dalam dan sekitar petak serta mengamati daya dukung terhadap vegetasi. Prosedur yang perlu dilakukan diantaranya:

\begin{itemize}
\tightlist
\item
  Catat beragam informasi lingkungan mencakup: tipe hutan, persentase tutupan bawah, hidrologi (ada/tidak sumber air dan berapa jauh dari petak), dan terrain system (tipe tanah, kemiringan, rawa, terendam permanen/temporer, dll.)
\item
  Informasi lain yang penting atau sekiranya berpengaruh terhadap dinamika ekosistem hutan dicatat pula, seperti adanya gangguan pembalakan, bekas kebakaran, dan lain-lain.
\item
  Ambil gambar dari rona lingkungan menggunakan kamera digital, seminimal nya foto empat arah (utara, timur, selatan, barat) di titik 0,0 (Gambar \ref{fig:figrlv}). Kondisi rona lingkungan yang menarik diambil pula gambarnya seperti bukaan di dalam hutan, bekas tebangan pohon, dan lain-lain.
\end{itemize}

\begin{figure}

{\centering \includegraphics[width=1\linewidth]{images/rlv_ilustration} 

}

\caption{Contoh gambar foto rona lingkungan}\label{fig:figrlv}
\end{figure}

\hypertarget{pengukuran-tutupan-tajuk}{%
\subsubsection*{Pengukuran tutupan tajuk}\label{pengukuran-tutupan-tajuk}}
\addcontentsline{toc}{subsubsection}{Pengukuran tutupan tajuk}

Tutupan kanopi sangat erat kaitannya dengan dinamika ekosistem di bawahnya seperti menyediakan iklim mikro bagi vegetasi dan habitat beragam satwa. Untuk mengukur tutupan tajuk, ukur persentase tutupan kanopi menggunakan densiometer atau kamera lensa \emph{fisheye}. Lakukan pengukuran di titik (10,0); (10,50); dan (10,100). Pendugaan tutupan tajuk seperti pada Gambar \ref{fig:figpkv}

\begin{figure}

{\centering \includegraphics[width=1\linewidth]{images/pkv_ilustration} 

}

\caption{Contoh pendugaan tutupan tajuk. (Sumber; https://doi.org/10.1016/j.uclim.2019.01.004)}\label{fig:figpkv}
\end{figure}

\hypertarget{pengukuran-tegakan-pohon}{%
\subsubsection*{Pengukuran tegakan pohon}\label{pengukuran-tegakan-pohon}}
\addcontentsline{toc}{subsubsection}{Pengukuran tegakan pohon}

Setelah petak selesai dibuat maka pengukuran tiap tegakan dapat dimulai. Pengukuran tiap tegakan mencakup diameter, tinggi, koleksi dan foto spesimen, hingga penandaan tegakan. Tegakan pohon yang diukur sesuai dengan kelas diameter dan anak-petaknya.

\hypertarget{pengukuran-diameter-pohon}{%
\paragraph*{Pengukuran diameter pohon}\label{pengukuran-diameter-pohon}}
\addcontentsline{toc}{paragraph}{Pengukuran diameter pohon}

Diameter pohon merupakan data penting untuk estimasi biomassa dan penguasaan lahan oleh jenis. Pengukuran diameter idealnya dilakukan pada batang utama pohon menggunakan diameter tape (phi-band) di ketinggian setinggi dada (\emph{diameter breast height} -- dbh). Untuk konsistensi titik pengukuran maka tinggi dbh ini ditetapkan 1,3m dari permukaan tanah. Untuk lebih memudahkan pengerjaan, buat tongkat kayu sepanjang 1,3m terlebih dahulu untuk mengukur tinggi 1,3m ini. Pengukuran tersebut dilakukan hanya bila bentuk batang di titik 1,3m tersebut bulat sempurna.

\begin{figure}

{\centering \includegraphics[width=1\linewidth]{images/dbhm_ilustration} 

}

\caption{Cara mengukur diameter setinggi dada secara langsung. (Sumber: http://www.epa.gov)}\label{fig:figdbhm}
\end{figure}

Dimensi bentuk pohon seringkali tidak membulat sempurna atau memiliki bentuk khusus seperti banir tinggi atau akar tunjang. Untuk pohon-pohon dengan kondisi khusus tersebut, pengukuran dilakukan dengan ketentuan seperti pada ilustrasi tabel \ref{tab:bkk} di bawah ini.

\begin{longtable}[]{@{}
  >{\raggedright\arraybackslash}p{(\columnwidth - 2\tabcolsep) * \real{0.5000}}
  >{\raggedright\arraybackslash}p{(\columnwidth - 2\tabcolsep) * \real{0.5000}}@{}}
\caption{\label{tab:bkk} Teknik-teknik pengukuran diameter batang dengan kondisi khusus}\tabularnewline
\toprule()
\begin{minipage}[b]{\linewidth}\raggedright
Keterangan
\end{minipage} & \begin{minipage}[b]{\linewidth}\raggedright
Ilustrasi
\end{minipage} \\
\midrule()
\endfirsthead
\toprule()
\begin{minipage}[b]{\linewidth}\raggedright
Keterangan
\end{minipage} & \begin{minipage}[b]{\linewidth}\raggedright
Ilustrasi
\end{minipage} \\
\midrule()
\endhead
\textbf{Pohon di tanah yang miring.} Pengukuran diameter pohon di tanah yang miring dimulai dengan mengukur tinggi 1,3m dari permukaan tanah yang paling tinggi menyentuh dasar pohon. DBH dihitung pada bagian batang di titik 1,3m tersebut & \includegraphics{images/vtb1.jpg} \\
\textbf{Pohon cacat (\emph{deform}) di titik 1,3 m.} Jika di titik 1,3m dari permukaan tanah, batang tidak bulat sempurna/cacat. maka titik pengukuran adalah pada 0,3m di atas dari ujung bagian yang cacat tersebut & \includegraphics{images/vtb2.jpg} \\
\textbf{Pohon miring (\emph{leaning}).} Pengukuran pada pohon miring dilakukan dengan mengukur 1,3m dari tanah pada sisi arah miring pohon tersebut/sisi yang terdekat dengan tanah & \includegraphics{images/vtb3.jpg} \\
\textbf{Pohon berbanir atau berakar tunjang.} jika tinggi banir atau akar tunjang di bawah 1,3m, maka pengukuran tetap dilakukan pada tinggi 1,3m & \includegraphics{images/vtb4.jpg} \\
Jika batas tinggi banir atau akar tunjang di atas 1,3 m dari permukaan tanah, maka pengukuran dilakukan di titik 0,3 m dari batas banir/akar tunjang & \includegraphics{images/vtb5.jpg} \\
Bila batas banir terlalu tinggi sehingga tidak memungkinkan untuk dipanjat, maka lakukan teknik dua tiang sesuai gambar di samping. Tiang didapat dari tegakan tingkat tiang lurus yang berasal dari luar petak dengan tinggi yang sesuai. Dua tiang disandingkan berjajar mengapit batang pada titik ukur diameter, orang ketiga memastikan bahwa tiang telah lurus, sejajar dan tepat di titik diameter, kemudian jarak antar tiang diukur dan dianggap sebagai diameter batang & \includegraphics{images/vtb6.jpg} \\
\textbf{Pohon Bercabang atau Pohon berbatang ganda.} Jika pohon bercabang dengan tinggi cabang pada atau di atas 1,3m (a) maka pengukuran diameter dilakukan di bawah titik percabangan. Jika cabang pohon berada di bawah 1,3m (b), maka masing-masing batang diukur pada ketinggian 1,3 m dari permukaan tanah dan masing-masing cabang dianggap 1 tegakan & \includegraphics{images/vtb7.jpg} \\
\textbf{Pohon tumbang.} Pohon yang tumbang namun masih tumbuh dedaun hidup dianggap sebagai pohon hidup. Pengukuran diameter pohon ini dilakukan pada bagian batang sejauh 1,3m dari batas akar dan batang & \includegraphics{images/vtb8.jpg} \\
\textbf{Pohon dengan liana.} Jika liana tumbuh pada titik pengukuran, jangan potong liana tersebut. Jika memungkinkan, Tarik liana menjauhi batang dan sematkan meter dbh tape di belakang liana sehingga dbh tape dapat melingkari batang dengan sempurna. Jika tidak memungkinkan, maka dengan bantuan pisau atau parang, buat dbh tape menyisip di antara liana sehingga dapat melingkari batang. Atau, jika liana sudah terlampau besar atau sudah sangat menempel pada batang, maka gunakan sisi belakang dari dbh tape (atau meteran ukur) untuk mengukur diameter batang secara visual & \includegraphics{images/vtb9.jpg} \\
\bottomrule()
\end{longtable}

\hypertarget{pengukuran-tinggi-pohon}{%
\paragraph*{Pengukuran tinggi pohon}\label{pengukuran-tinggi-pohon}}
\addcontentsline{toc}{paragraph}{Pengukuran tinggi pohon}

Tinggi pohon yang diukur adalah tinggi bebas cabang (TBC) dan tinggi total (TT) menggunakan laser distometer/distancemeter. Konsep penggunaan distometer adalah merekam secara langsung jarak laser dari alat ke target yang dituju. Cara penggunaan distometer adalah arahkan laser dari dasar pohon secara langsung ke arah pucuk pohon (untuk mengukur tinggi total) atau ke arah cabang pertama (untuk mengukur tinggi bebas cabang) kemudian rekam jaraknya. Cara tersebut dapat digunakan bila kondisi medan memungkinkan untuk laser mencapai target tanpa terhalang oleh apapun sehingga tinggi dapat langsung direkam. Pada kondisi sebenarnya, kondisi tersebut hampir tidak memungkinkan, karena banyaknya penghalang di dalam hutan sehingga laser tidak dapat mencapai target. Cara yang dapat digunakan adalah menggunakan konsep pytagoras, dengan cara mengambil jarak dari pohon target sejauh mana laser dapat mencapai target. Kemudian setting distometer ke mode pengukuran pitagoras. Dari jarak tersebut, arahkan pointer laser ke pohon target pada tiga arah: cabang pohon, searah tinggi distometer, dan dasar pohon seperti terlihat pada gambar \ref{fig:figdst}, kemudian rekam. Distometer secara otomatis akan menghitung tinggi pohon secara langsung.

\begin{figure}

{\centering \includegraphics[width=1\linewidth]{images/dst_ilustration} 

}

\caption{Alat distometer dan konsep pengukuran tinggi pohon. (Sumber:theoriginallaserdistancemeter.co.uk)}\label{fig:figdst}
\end{figure}

\textbf{Hal yang harus diperhatikan:}

Pengukuran langsung terhadap target tinggi total besar kemungkinan sulit dilakukan karena kendala banyaknya penghalang di sekitar target. Dalam kondisi demikian, penggunaan laser distometer dimaksudkan untuk meminimalisir margin eror dalam pengukuran tinggi secara langsung. Arahkan target laser ke bagian yang sedekat mungkin dengan puncak target, kemudian dari jarak tersebut tambahkan estimasi tinggi hingga ke puncak. Estimasi yang dibuat tersebut diasumsikan sebagai margin eror dari tinggi sebenarnya.

\hypertarget{foto-dan-koleksi-spesimen-1}{%
\paragraph*{Foto dan koleksi spesimen}\label{foto-dan-koleksi-spesimen-1}}
\addcontentsline{toc}{paragraph}{Foto dan koleksi spesimen}

Identifikasi di lapangan sering kali sangat sulit dilakukan, terlebih bila jenis tersebut belum pernah dijumpai sebelumnya. Dalam pencatatan, nama pohon teridentifikasi di lapangan bisa berupa nama lokal dan/atau tingkat taksa yang paling mendekati, seperti misal Fabaceae, Shorea sp. Untuk identifikasi lebih lanjut, spesimen setiap jenis harus diambil.

Pengambilan spesimen sedapat mungkin lengkap mewakili karakter penting untuk indetifikasi. Karakter tersebut diantaranya daun dengan ranting-rantingnya, bunga dan buah, serta karakter khusus lainnya seperti damar. Upayakan untuk sedapat mungkin mengoleksi spesimen tersebut untuk dipreparasi atau herbaria. Jika spesimen karakter-karakter tersebut sangat sulit untuk dicuplik, maka seminimalnya ambil serasah daunnya paling sedikit 3 lembar per jenis kemudian beri nama kode pohon dengan spidol marker. Teknik preparasi spesimen herbaria dapat dilihat pada \sout{lampiran xx}. Catat ciri-ciri khas dari jenis di bagian keterangan pada lembar data, seperti berdamar susu, berduri panjang, dll.

Selain pengambilan spesimen, pengambilan gambar dari karakter tumbuhan pun diperlukan untuk melengkapi dan membantu proses identifikasi. Teknik pengambilan gambar karakter tersebut sebisa mungkin mewakili setiap karakter yang dibutuhkan untuk proses identifikasi. Karakter yang wajib diambil gambarnya mencakup: daun dan struktur percabangannya, batang lengkap dengan takikan batangnya, karakter generatif (bila dijumpai), serta karakter khusus seperti damar, getah, dan duri. Contoh pengambilan gambar karakter tersebut dapat dilihat pada tabel \ref{tab:tabgkj}

\textbf{Hal yang perlu diperhatikan:}

Daun serasah yang dicuplik harus tepat berasal dari pohon jenis yang dimaksud dan tidak dalam kondisi rusak/terdekomposisi. Gunakan teropong atau bantuan lensa zoom dari kamera untuk memastikan bentuk daun yang masih berada di atas pohon.

\begin{longtable}[]{@{}
  >{\raggedright\arraybackslash}p{(\columnwidth - 2\tabcolsep) * \real{0.5000}}
  >{\raggedright\arraybackslash}p{(\columnwidth - 2\tabcolsep) * \real{0.5000}}@{}}
\caption{\label{tab:tabgkj} Contoh-contoh pengambilan gambar karakter jenis}\tabularnewline
\toprule()
\begin{minipage}[b]{\linewidth}\raggedright
Keterangan
\end{minipage} & \begin{minipage}[b]{\linewidth}\raggedright
Ilustrasi
\end{minipage} \\
\midrule()
\endfirsthead
\toprule()
\begin{minipage}[b]{\linewidth}\raggedright
Keterangan
\end{minipage} & \begin{minipage}[b]{\linewidth}\raggedright
Ilustrasi
\end{minipage} \\
\midrule()
\endhead
\textbf{Batang utama.} Informasi yang diharapkan: warna batang, tekstur permukaan kulit, ada tidaknya banir/akar tanjung. & \includegraphics{images/gkj1.jpg} \\
\textbf{Daun hidup.} setidaknya harus mencakup informasi percabangan, bentuk, warna, dan tipe formasi duduk daun & \includegraphics{images/gkj2.jpg} \\
\textbf{Takikan batang dan daun serasah.} Untuk mengidentifikasi dari kulit luar, kulit dalam, urat kayu, dan ciri khusus seperti getah. Informasi mengenai warna masing-masing karakter sangat penting untuk menentukan jenis & \includegraphics{images/gkj3.jpg} \\
\textbf{Organ generatif.} Bila dijumpai bunga dan atau buah dari jenis, ambil gambarnya setidaknya gambar utuh. Sertakan pembanding saat pengambilan gambar & \includegraphics{images/gkj4.jpg} \\
\textbf{Ciri - ciri khusus.} Bila jenis memiliki ciri khusus seperti getah, damar, duri, dan lain sebagainya, ambil gambar dari ciri khusus tersebut & \includegraphics{images/gkj5.jpg} \\
\bottomrule()
\end{longtable}

\hypertarget{penandaan-pohon}{%
\paragraph*{Penandaan pohon}\label{penandaan-pohon}}
\addcontentsline{toc}{paragraph}{Penandaan pohon}

Penandaan pohon dilakukan dengan maksud selain memberikan ID setiap tegakan yang diukur, juga untuk memudahkan pencarian dalam pengukuran ulang atau verifikasi di masa mendatang. Penandaan dilakukan dengan memasang plat alumunium berukuran kurang lebih 6 x 3 cm yang memuat setidaknya informasi kode pohon. Pemasangan alumunium dilakukan dengan menancapkan plat pada pohon dengan paku. Contoh pemasangan dapat dilihat pada gambar \ref{fig:figtag}.

\begin{figure}

{\centering \includegraphics[width=1\linewidth]{images/tagging} 

}

\caption{Contoh pemasangan plat tagging}\label{fig:figtag}
\end{figure}

\hypertarget{analisa-data}{%
\chapter*{Analisa Data}\label{analisa-data}}
\addcontentsline{toc}{chapter}{Analisa Data}

\hypertarget{pengelolaan-data}{%
\chapter*{Pengelolaan Data}\label{pengelolaan-data}}
\addcontentsline{toc}{chapter}{Pengelolaan Data}

\hypertarget{lampiran-1.-lembar-data}{%
\chapter*{Lampiran 1. Lembar Data}\label{lampiran-1.-lembar-data}}
\addcontentsline{toc}{chapter}{Lampiran 1. Lembar Data}

Peran lembar data dalam kajian survei kehati sangat penting. Penggunaan lembar data yang tepat membuat pencatatan temuan menjadi lebih efisien dan terstandarisasi. Dalam lampiran ini terlampir contoh-contoh lembar data untuk setiap taksa. Templat lembar data tersedia pada tautan ini: \emph{Tallysheet-biodive}. Pembaca bisa mengunduh dan memperbanyak lembar data sebanyak yang dibutuhkan sebelum survei. Pada praktiknya, mungkin lembar data yang penulis sediakan belum mencakup hal spesifik yang dibutuhkan oleh projek, oleh karena itu pembaca bisa menambahkan sendiri kolom-kolom yang dibutuhkan.

Selalu gunakan pensil dalam menulis di lembar data dan jika memungkinkan gunakan kertas tahan air, karena kemungkinan basah karena hujan sangat tinggi. Setelah selesai dari lapang, harus langsung dipindai untuk disimpan sebagai salinan digital. Lembar data yang ditulis dilapangan ini merupakan data primer untuk verifikasi seandainya ada kesalahan input saat surveior memindahkan ke dalam excel.

\hypertarget{lembar-data-avifauna}{%
\section*{Lembar Data Avifauna}\label{lembar-data-avifauna}}
\addcontentsline{toc}{section}{Lembar Data Avifauna}

\textbf{Lembar data pengamatan menggunakan titik hitung}

Pada awal lembar data dibutuhkan informasi umum mengenai tanggal, lokasi, durasi pengamatan, dan seluruh personil yang terlibat. Untuk lokasi geografis dari GPS, set menjadi decimal degree supaya bisa konsisten diseluruh Indonesia dan mudah di-input ke dalam sistem computer (Excel, dll). Keterangan dari setiap kolom adalah sebagai berikut;

\textbf{Jenis:} Nama jenis burung menggunakan nama latin, namun apabila belum mengetahui jenisnya, dapat menggunakan nama lokal terlebih dahulu.

\textbf{Individu:} Jumlah burung yang ditemukan pada satu spot (beberapa burung terkadang berkelompok atau berpasangan seperti cendrawasih atau burung gereja)

\textbf{Jarak:} Jarak burung dari pengamat dalam satuan meter

\textbf{Catatan:} Tambahan catatan penting jika ada

\begin{figure}

{\centering \includegraphics[width=1\linewidth]{images/ldp_th} 

}

\caption{Contoh lembar data untuk metode titik hitung}\label{fig:ldpth}
\end{figure}

\textbf{Lembar data parameter lingkungan di titik hitung}

Dalam setiap titik hitung, dapat ditambahkan parameter lingkungan untuk melihat pengaruh perbedaan rona lingkungan terhadap komunitas burung, Adapun keterangan dari setiap baris adalah sebagai berikut

\textbf{Tallest tree (m):} Pohon tertinggi disekitar lokasi pengamatan. Satuan nilai dalam meter

\textbf{Ground cover (\%):} Tutupan bawah disekitar lokasi pengamatan satuan nilai (\%)

\textbf{Plant height 0-1,5 m (\%):} Jumlah persentase pohon dengan ukuran 0-1.5 m disekitar lokasi pengamatan

\textbf{Plant height 1,5-5 m (\%):} Jumlah persentase pohon dengan ukuran 1.5-5 m disekitar lokasi pengamatan

\textbf{Plant height 5-15 m (\%):} Jumlah persentase pohon dengan ukuran 5-15 m disekitar lokasi pengamatan

\textbf{Plant height \textgreater15 m (\%):} Jumlah persentase pohon dengan ukuran \textgreater15 m disekitar lokasi pengamatan

\textbf{Bole climb (\%):} Persentase tumbuhan yang merambat disekitar lokasi pengamatan

\textbf{Liana (\%):} Persentase tumbuhan pemanjat disekitar lokasi pengamatan

\textbf{Macaranga (\%):} Persentase tumbuhan jenis macaranga disekitar lokasi pengamatan

\textbf{Rattan (\%):} Persentase rotan disekitar lokasi pengamatan

\textbf{Fern (\%):} Persentase paku-pakuan disekitar lokasi pengamatan

\textbf{Small palm (\%):} Persentase palem-paleman disekitar lokasi pengamatan

\textbf{Dist. from water:} Kategori jarak ke sumber air; 1 = 0-50 m, 2 = 50-100m, 3 \textgreater{} 100m

\textbf{Logs abd:} Jumlah pohon tumbang yang ada dilokasi pengamatan

\textbf{Snags abd:} Jumlah pohon mati berdiri disekitar lokasi pengamatan

\textbf{Zingiberaceae (\%):} Persentase temu-temuan/rimpang disekitar lokasi pengamatan

\textbf{Grass (\%):} Persentase rumput-rumputan disekitar lokasi pengamatan

\textbf{Moss (cm):} Ketebalan lumut disekitar lokasi pengamatan dalam centimeter

\textbf{Litter (cm):} Ketebalan seresah disekitar lokasi pengamatan dalam centimeter

\begin{figure}

{\centering \includegraphics[width=1\linewidth]{images/ldp_pl} 

}

\caption{Contoh lembar data untuk parameter lingkungan di setiap titik hitung}\label{fig:ldppl}
\end{figure}

\textbf{Lembar data daftar jenis MacKinnon}

Pada pengamatan yang bersifat ekploratif menggunakan daftar jenis MacKinnon, lembar data yang digunakan sangat sederhana, dengan diawali informasi pengamat, lokasi dan durasi pengamatan

\begin{figure}

{\centering \includegraphics[width=1\linewidth]{images/ldp_ml} 

}

\caption{Contoh lembar data untuk daftar jenis MacKinnon}\label{fig:ldpml}
\end{figure}

\hypertarget{lembar-data-herpetofauna}{%
\section*{Lembar Data Herpetofauna}\label{lembar-data-herpetofauna}}
\addcontentsline{toc}{section}{Lembar Data Herpetofauna}

\textbf{Lembar data pengamatan menggunakan metode \emph{VES}}

Pada awal lembar data dibutuhkan informasi umum mengenai tanggal, lokasi, durasi pengamatan, dan seluruh personil yang terlibat. Untuk lokasi geografis dari GPS, set menjadi decimal degree supaya bisa konsisten diseluruh Indonesia dan mudah di-input ke dalam sistem computer (Excel, dll). Keterangan dari setiap kolom adalah sebagai berikut;

\textbf{Waktu:} Jam ditemukannya herpetofauna, gunakan format hh;mm (0:00 -- 24:00)

\textbf{Jenis:} Nama jenis menggunakan nama latin, namun apabila belum mengetahui jenisnya, dapat menggunakan nama lokal terlebih dahulu.

\textbf{SVL (cm):} Panjang tubuh dari moncong hingga pangkal ekor dalam satuan cm

\textbf{Hor (m):} (a) survei di sempadan sungai; jarak horisontal dari sungai (b) survei di transek; jarak horisontal dari garis tengah transek

\textbf{Ver (m):} (a) survei di sempadan sungai; Jarak vertikal dari badan air (b) survei di transek; jarak vertikal dari permukaan tanah

\textbf{Substrat:} Substrat atau pijakan dari satwa yang ditemukan

\textbf{Aktivitas:} Aktivitas spesifik pada saat ditemukan

\begin{figure}

{\centering \includegraphics[width=1\linewidth]{images/ldh_ves} 

}

\caption{Contoh lembar data untuk metode VES}\label{fig:ldhves}
\end{figure}

\hypertarget{lembar-data-mamalia}{%
\section*{Lembar Data Mamalia}\label{lembar-data-mamalia}}
\addcontentsline{toc}{section}{Lembar Data Mamalia}

\textbf{Lembar data pengamatan menggunakan transek garis dan eksplorasi}

Pada awal lembar data dibutuhkan informasi umum mengenai tanggal, lokasi, durasi pengamatan, dan seluruh personil yang terlibat. Untuk lokasi geografis dari GPS, set menjadi decimal degree supaya bisa konsisten diseluruh Indonesia dan mudah di-input ke dalam sistem computer (Excel, dll). Keterangan dari setiap kolom adalah sebagai berikut;

\textbf{Waktu:} Jam ditemukannya mamalia, gunakan format hh;mm (0:00 -- 24:00)

\textbf{Jenis:} Nama jenis menggunakan nama latin, namun apabila belum mengetahui jenisnya, dapat menggunakan nama lokal terlebih dahulu.

\textbf{PPD (M):} Jarak perpendicular satwa ditemukan pertama kali, relatif terhadap garis tengah transek jika memungkinkan.

\textbf{Tipe temuan:} Tipe temuan satwa dapat diisi dengan perjumpaan langsung, kotoran, cakaran, maupun suara. Pembeda temuan ini untuk mengukur akurasi identifikasi.

\textbf{GPS ID:} Nomor waypoint pada GPS. Supaya survei lebih efisien, pengamat boleh menulis nomor waypoint pada GPS terlebih dahulu selama pengamatan, untuk kemudian melengkapi koordinat pada saat sudah di kamp.

\textbf{Lon:} Lokasi koordinat Longitude (X). Referensi kordinat yang dipakai adalah WGS84 dengan format penulisan decimal degree.

\textbf{Lat:} Lokasi koordinat Latitude (Y). Referensi kordinat yang dipakai adalah WGS84 dengan format penulisan decimal degree.

\textbf{Catatan:} Tambahkan catatan penting jika ada

\begin{figure}

{\centering \includegraphics[width=1\linewidth]{images/ldm_tg} 

}

\caption{Contoh lembar data untuk metode transek garis}\label{fig:ldmtg}
\end{figure}

\textbf{Lembar data pengamatan mamalia kecil menggunakan perangkap}

Khusus untuk pengamatan mamalia kecil menggunakan perangkap, menggunakan contoh lembar data dan keterangan di bawah ini. Lembar data ini terdiri dari dua bagian, dimana bagian kedua berisi informasi perangkap yang digunakan pada bagian selanjutnya untuk dapat mengukur usaha survei dan informasi lokasi disetiap perangkap.

\textbf{Tanggal:} Tanggal ditemukan satwa

\textbf{Waktu:} Jam ditemukan satwa

\textbf{Trap ID:} Nomor unik setiap perangkap yang digunakan

\textbf{Jenis:} Nama jenis menggunakan nama latin, namun apabila belum mengetahui jenisnya, dapat menggunakan nama lokal terlebih dahulu.

\textbf{Usia:} kriteria usia satwa jika diketahui

\textbf{Sex:} Jenis kelamin satwa jika diketahui

\textbf{HB (mm):} Panjang tubuh satwa dalam satuan milimeter

\textbf{FA (mm):} Panjang lengan atau kaki depan satwa dalam sattuan milimeter

\textbf{E (mm):} Panjang telinga satwa

\textbf{T (mm):} Panjang ekor satwa

\textbf{HF (mm): }Panjang kaki belakang satwa

\textbf{W (gr):} Berat tubuh satwa dalam gram

\textbf{AT (mm): }Panjang anti tragus dalam milimeter, khusus untuk satwa kelelawar

\textbf{TR (mm):} Panjang tragus dalam milimeter, khusus untuk satwa kelelawar

\textbf{Catatan: }Tambahan catatan penting jika ada

\begin{figure}

{\centering \includegraphics[width=1\linewidth]{images/ldm_pk} 

}

\caption{Contoh lembar data untuk pengamatan menggunakan perangkap}\label{fig:ldmpk}
\end{figure}

\textbf{Lembar data informasi perangkap yang digunakan}

Lembar data dibawah ini merupakan bagian kedua yang berisi informasi seluruh perangkap selama melakukan kajian. Lembar data ini merupakan bagian kedua dari lembar data sebelumnya, dengan informasi setiap kolom sebagai berikut;

\textbf{Trap ID:} Nomor unik setiap perangkap yang digunakan

\textbf{Tipe:} Tipe atau jenis perangkap yang digunakan

\textbf{GPS ID:} Nomor waypoint pada GPS. Supaya survei lebih efisien, pengamat boleh menulis nomor waypoint pada GPS terlebih dahulu selama pengamatan, untuk kemudian melengkapi koordinat pada saat sudah di kamp.

\textbf{Lon:} Lokasi koordinat Longitude (X). Referensi kordinat yang dipakai adalah WGS84 dengan format penulisan decimal degree.

\textbf{Lat:} Lokasi koordinat Latitude (Y). Referensi kordinat yang dipakai adalah WGS84 dengan format penulisan decimal degree.

\textbf{Tanggal pasang:} Tanggal perangkap mulai dipasang atau diaktifkan

\textbf{Tangggal selesai:} Tanggal perangkap selesai digunakan

\textbf{Waktu pasang:} Waktu mulai perangkap dipasang atau diaktifkan

\textbf{Waktu selesai:} waktu berakhirnya perangkap digunakan

\textbf{Catatan:} Tambahan catatan penting jika ada

\begin{figure}

{\centering \includegraphics[width=1\linewidth]{images/ldm_p} 

}

\caption{Contoh lembar data informasi perangkap}\label{fig:ldmp}
\end{figure}

\hypertarget{lembar-data-vegetasi}{%
\section*{Lembar Data Vegetasi}\label{lembar-data-vegetasi}}
\addcontentsline{toc}{section}{Lembar Data Vegetasi}

\textbf{Lembar data informasi petak vegetasi}

Lembar data vegetasi memiliki dua bagian utama. Bagian pertama berisi mengenai informasi personil, lokasi referensi geografis, dan keadaan umum disekitar plot dengan contoh dan informasi dibawah ini;

\textbf{Nomor ID Plot:} Nomor unik plot

\textbf{Koordinat GPS (X):} Lokasi koordinat Longitude (X). Referensi kordinat yang dipakai adalah WGS84 dengan format penulisan decimal degree.

\textbf{Koordinat GPS (Y):} Lokasi koordinat Latitude (Y). Referensi kordinat yang dipakai adalah WGS84 dengan format penulisan decimal degree.

\textbf{Arah plot:} arah plot (barat, timur, utara, selatan)

\textbf{Waktu (tanggal, jam):} informasi waktu dan jam memulai pengamatan di plot

\textbf{Cuaca:} cuaca pada saat pengamatan

\textbf{Substrat:} Kondisi subsrat dominan yang ada di plot

\textbf{Anggota tim:} nama-nama seluruh personil

\textbf{Tipe habitat:} Kondisi deksriptif habitat atau vegetasi yang dominan di plot

\textbf{Drainase lantai hutan:} Kondisi lantai hutan, tandai dengan silang pilihan yang ada disebelah kanannya

\textbf{Gangguan:} Gangguan atau potensi gangguan yang ada di dalam dan sekitar plot

\textbf{Nomor foto rona vegetasi:} nomor foto yang ada di kamera, mengenai foto-foto rona vegetasi yang ada di dalam plot

\textbf{Catatan tambahan:} Catatan tambahan jika diperlukan mengenai kondisi plot.

\begin{figure}

{\centering \includegraphics[width=1\linewidth]{images/ldv_ip} 

}

\caption{Contoh lembar data petak vegetasi}\label{fig:ldvip}
\end{figure}

\textbf{Lembar data vegetasi}

Lembar data ini merupakan bagian kedua dari set lembar data vegetasi untuk menulis pengukuran tumbuhan di setiap anak petak dengan contoh dan informasi setiap kolom sebagai berikut;

\textbf{Kelas:} Kelas kategori tumbuhan berdasarkan ukuran diameter batang

\textbf{Jenis:} Nama jenis menggunakan nama latin, namun apabila belum mengetahui jenisnya, dapat menggunakan nama lokal terlebih dahulu.

\textbf{Kode pohon:} Kode penanda individu tumbuhan didalam plot

\textbf{DBH (cm):} Diameter batang setinggi dada (Dbh) dalam satuan cm

\textbf{TT (m):} Tinggi total tanaman dalam meter

\textbf{TBC (m):} Tinggi bebas cabang dalam satuan meter

\textbf{Kode foto:} Nomor foto di kamera
\textbf{Keterangan:} Catatan atau keterangan tambahan jika diperlukan

\begin{figure}

{\centering \includegraphics[width=1\linewidth]{images/ldv_m} 

}

\caption{Contoh lembar data vegetasi}\label{fig:ldvm}
\end{figure}

\hypertarget{lampiran-2.-teknik-preservasi-spesimen}{%
\chapter*{Lampiran 2. Teknik Preservasi Spesimen}\label{lampiran-2.-teknik-preservasi-spesimen}}
\addcontentsline{toc}{chapter}{Lampiran 2. Teknik Preservasi Spesimen}

Seringkali spesies yang ditemukan saat pengamatan belum bisa teridentifikasi hingga ke tingkat \emph{species}. Ketika hal ini terjadi, umumnya pengamat harus menangkap jenis tersebut untuk diawetkan sebagai spesimen awetan (\emph{voucher specimens}). Hal ini menjadi penting untuk dilakukan karena: (1) Memastikan identifikasi jenis secara akurat dan, (2) Meningkatkan pemahaman terhadap kemungkinan variasi jenis dari lokasi yang berbeda. Pengawetan spesimen ini juga dapat digunakan dikemudian hari oleh peneliti lain seperti studi DNA, revisi taksonomi dan sebagainya.

Meskipun penting untuk dilakukan, perlu diperhatikan juga mengenai etika dalam koleksi spesimen awetan. Beberapa kawasan konservasi mungkin membutuhkan legalitas yang harus diurus sebelum dapat membawa spesimen keluar kawasan. Ketika dikawasan non-konservasi pun kita harus sadar terhadap kerentanan spesies tersebut dan dampak yang akan terjadi terhadap pengambilan spesies untuk dijadikan spesimen awetan. Jika anda menangkap dan membunuh banyak hewan untuk alasan penelitian, justru menimbulkan dampak yang negatif terhadap upaya konservasi yang sedang anda upayakan. Oleh karena itu, perijinan untuk pengambilan spesimen dan tempat penyimpanan spesimen (laboratorium zoologi atau herbarium di universitas terdekat) harus didapatkan terlebih dahulu sebelum melakukan kajian, supaya spesies yang sudah ditangkap tidak mati sia-sia dan dapat digunakan sebaik-baiknya untuk penelitian.

\hypertarget{herpetofauna-1}{%
\section*{Herpetofauna}\label{herpetofauna-1}}
\addcontentsline{toc}{section}{Herpetofauna}

Data yang diperlukan pada spesimen awetan herpetofauna harus memiliki catatan nama jenis, lokasi pengambilan sampel, tanggal dan waktu, kode sampel, berat, ukuran panjang, dan hal lain yang terkait.

\textbf{Tahapan pengawetan:}

\begin{itemize}
\tightlist
\item
  Untuk keamanan gunakan sarung tangan lateks dan masker saat melakukan kegiatan preservasi.
\item
  Lakukan pembiusan dengan menekan kapas yang sebelumnya telah dicelupkan ke cairan MS-222 atau Chlorobutanol pada lubang pernafasan. Saat larutan MS-222 atau Chlorobutanol tidak dapat diperoleh maka dapat diganti dengan larutan alkohol 70\%.
\item
  Setelah dipastikan sampel terbius (terlihat lemas), lakukan penyuntikkan alcohol 70\% menggunakan \emph{syringe} pada otak kecil melalui tengkuk.
\item
  Lakukan penyuntikan formalin 10\% pada bagian-bagian berdaging, berongga dan organ dalam.
\item
  Segera setelah penyuntikkan, letakkan sampel tersebut dalam suatu kotak plastik yang sebelumnya telah dilapisi kain kasa dan dibasahi oleh larutan formalin 10\%.
\item
  Atur posisi spesimen dengan posisi menunjukkan morfologi spesimen terlihat untuk memudahkan identifikasi ulang di laboratorium jika dibutuhkan. Misalnya mulut disumpal dengan kapas untuk menunjukkan bagian dalam mulut, jari-jari tungkai dimekarkan untuk melihat selaput (gambar \ref{fig:spesamf1}). Kemudian ikatkan label pada bagian pinggang agar tidak tertukar dengan spesimen lain.
\item
  Tutupi spesimen dengan kain yang dibasahi formalin 10\% dan tutup box selama satu sampai dua hari hingga spesimen kaku.
\item
  Untuk penyimpanan permanen, spesimen yang telah dipreservasi dalam kotak plastik dikeluarkan dan dibilas dengan air mengalir selama 1 -- 2 jam untuk menghilangkan sisa formalin. Setelah itu dapat dipindahkan ke dalam toples kaca yang berisi alkohol 70\%
\end{itemize}

\begin{figure}

{\centering \includegraphics[width=1\linewidth]{images/spesimenafmibi1} 

}

\caption{Foto spesimen Amfibi; bagian lateral (a), bagian dorsal (b), bagian ventral (c).}\label{fig:spesamf1}
\end{figure}

\begin{figure}

{\centering \includegraphics[width=1\linewidth]{images/spesimenafmibi2} 

}

\caption{Spesimen Amfibi yang telah diatur posisinya dan diberi label spesimen untuk memudahkan identifikasi ulang di laboratorium}\label{fig:spesamf2}
\end{figure}

\hypertarget{mamalia-1}{%
\section*{Mamalia}\label{mamalia-1}}
\addcontentsline{toc}{section}{Mamalia}

Pengawetan satwa mamalia kecil dan kelelawar untuk identifikasi lebih lanjut atau koleksi spesimen adalah sebagai berikut:

\begin{itemize}
\tightlist
\item
  Lakukan pembiusan dengan menekan kapas yang sebelumnya telah dicelupkan ke cairan klorofom atau alkohol 70\%
\item
  Setelah dipastikan sampel terbius (terlihat lemas), lakukan penyuntikkan alcohol 70\% menggunakan \emph{syringe} pada otak kecil melalui tengkuk.
\item
  Lakukan pengambilan foto sebagai dokumentasi warna asli sebelum diawetkan
\item
  Sumbat mulut dengan kapas, supaya bagian mulut dapat diperiksa sebelum kaku dikemudian hari, lalu lakukan penyuntikan formalin 5-10\% pada bagian-bagian berdaging, berongga dan organ dalam.
\item
  Letakkan sampel tersebut ke dalam suatu kotak plastik yang sebelumnya telah dilapisi kain kasa dan dibasahi oleh larutan formalin 10\%, kemudian ikatkan label pada bagian kaki agar tidak tertukar dengan spesimen lain
\item
  Untuk penyimpanan permanen, spesimen yang telah dipreservasi dalam kotak plastik dikeluarkan dan dibilas dengan air mengalir selama 1 -- 2 jam untuk menghilangkan sisa formalin. Setelah itu dapat dipindahkan ke dalam toples kaca yang berisi alkohol 70\%
\end{itemize}

\hypertarget{vegetasi-1}{%
\section*{Vegetasi}\label{vegetasi-1}}
\addcontentsline{toc}{section}{Vegetasi}

Pengawetan tumbuhan dilakukan dengan tahapan sebagai berikut:

\begin{enumerate}
\def\labelenumi{\arabic{enumi}.}
\tightlist
\item
  Gunting bagian ranting tumbuhan yang memiliki daun lengkap dan tidak rusak (jika ranting memiliki bunga dan buah akan lebih baik) dengan panjang minimal 30 cm, tidak lupa berikan label (etiket) untuk penanda spesies
\end{enumerate}

\begin{figure}

{\centering \includegraphics[width=1\linewidth]{images/ranting} 

}

\caption{Bagian tumbuhan seteleah digunting 30 cm}\label{fig:ranting}
\end{figure}

\begin{enumerate}
\def\labelenumi{\arabic{enumi}.}
\setcounter{enumi}{1}
\item
  Semprotkan alkohol 70\% keseluruh bagian yang sudah dipotong untuk mencegah tumbuhnya jamur yang dapat merusak bagian tumbuhan
\item
  Bungkus tumbuhan dengan koran dan sempotkan kembali alkohol 70\% dan masukan ke dalam plastik untuk dibawa keluar
\item
  Setelah sampai di kamp, keluarkan ranting dari plastik, semprotkan kembali dengan alkohol 70\% dan ganti koran pembungkus dengan yang baru
\end{enumerate}

\begin{figure}

{\centering \includegraphics[width=1\linewidth]{images/preherba} 

}

\caption{Membungkus ranting dengan kertas koran dan pengepresan dengan triplek}\label{fig:preherba}
\end{figure}

\begin{enumerate}
\def\labelenumi{\arabic{enumi}.}
\setcounter{enumi}{4}
\tightlist
\item
  Ranting yang sudah dibungkus dengan koran kemudian di jepit dengan triplek atau papan kardus dan dimasukan ke oven dengan suhu 60 derjat celcius selama kurang lebih 3x24 jam
\end{enumerate}

\begin{figure}

{\centering \includegraphics[width=1\linewidth]{images/oven} 

}

\caption{Pengeringan dengan menggunakan oven}\label{fig:oven}
\end{figure}

\begin{enumerate}
\def\labelenumi{\arabic{enumi}.}
\setcounter{enumi}{5}
\tightlist
\item
  Setelah kering, ranting tumbuhan dijahit atau di tempel ke kertas karton dan berikan keterangan yang berisi Nama kolektor, Tanggal dan Lokasi tumbuhan ditemukan, nomor/ kode spesimen, Habitat dan deskripsi mengenai tumbuhan.
\end{enumerate}

\begin{figure}

{\centering \includegraphics[width=1\linewidth]{images/herba} 

}

\caption{Herbarium yang sudah kering dan siap disimpan }\label{fig:herba}
\end{figure}

\begin{enumerate}
\def\labelenumi{\arabic{enumi}.}
\setcounter{enumi}{6}
\tightlist
\item
  Untuk penyimpanan permanen, letakan herbarium pada kotak kedap udara atau tempat yang kering dan jemurlah herbarium sekali-kali dibawah sinar matahari.
\end{enumerate}

\end{document}
